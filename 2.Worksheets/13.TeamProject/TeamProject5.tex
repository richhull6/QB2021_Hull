\PassOptionsToPackage{unicode=true}{hyperref} % options for packages loaded elsewhere
\PassOptionsToPackage{hyphens}{url}
%
\documentclass[
]{article}
\usepackage{lmodern}
\usepackage{amssymb,amsmath}
\usepackage{ifxetex,ifluatex}
\ifnum 0\ifxetex 1\fi\ifluatex 1\fi=0 % if pdftex
  \usepackage[T1]{fontenc}
  \usepackage[utf8]{inputenc}
  \usepackage{textcomp} % provides euro and other symbols
\else % if luatex or xelatex
  \usepackage{unicode-math}
  \defaultfontfeatures{Scale=MatchLowercase}
  \defaultfontfeatures[\rmfamily]{Ligatures=TeX,Scale=1}
\fi
% use upquote if available, for straight quotes in verbatim environments
\IfFileExists{upquote.sty}{\usepackage{upquote}}{}
\IfFileExists{microtype.sty}{% use microtype if available
  \usepackage[]{microtype}
  \UseMicrotypeSet[protrusion]{basicmath} % disable protrusion for tt fonts
}{}
\makeatletter
\@ifundefined{KOMAClassName}{% if non-KOMA class
  \IfFileExists{parskip.sty}{%
    \usepackage{parskip}
  }{% else
    \setlength{\parindent}{0pt}
    \setlength{\parskip}{6pt plus 2pt minus 1pt}}
}{% if KOMA class
  \KOMAoptions{parskip=half}}
\makeatother
\usepackage{xcolor}
\IfFileExists{xurl.sty}{\usepackage{xurl}}{} % add URL line breaks if available
\IfFileExists{bookmark.sty}{\usepackage{bookmark}}{\usepackage{hyperref}}
\hypersetup{
  pdfborder={0 0 0},
  breaklinks=true}
\urlstyle{same}  % don't use monospace font for urls
\usepackage[margin=1in]{geometry}
\usepackage{color}
\usepackage{fancyvrb}
\newcommand{\VerbBar}{|}
\newcommand{\VERB}{\Verb[commandchars=\\\{\}]}
\DefineVerbatimEnvironment{Highlighting}{Verbatim}{commandchars=\\\{\}}
% Add ',fontsize=\small' for more characters per line
\usepackage{framed}
\definecolor{shadecolor}{RGB}{248,248,248}
\newenvironment{Shaded}{\begin{snugshade}}{\end{snugshade}}
\newcommand{\AlertTok}[1]{\textcolor[rgb]{0.94,0.16,0.16}{#1}}
\newcommand{\AnnotationTok}[1]{\textcolor[rgb]{0.56,0.35,0.01}{\textbf{\textit{#1}}}}
\newcommand{\AttributeTok}[1]{\textcolor[rgb]{0.77,0.63,0.00}{#1}}
\newcommand{\BaseNTok}[1]{\textcolor[rgb]{0.00,0.00,0.81}{#1}}
\newcommand{\BuiltInTok}[1]{#1}
\newcommand{\CharTok}[1]{\textcolor[rgb]{0.31,0.60,0.02}{#1}}
\newcommand{\CommentTok}[1]{\textcolor[rgb]{0.56,0.35,0.01}{\textit{#1}}}
\newcommand{\CommentVarTok}[1]{\textcolor[rgb]{0.56,0.35,0.01}{\textbf{\textit{#1}}}}
\newcommand{\ConstantTok}[1]{\textcolor[rgb]{0.00,0.00,0.00}{#1}}
\newcommand{\ControlFlowTok}[1]{\textcolor[rgb]{0.13,0.29,0.53}{\textbf{#1}}}
\newcommand{\DataTypeTok}[1]{\textcolor[rgb]{0.13,0.29,0.53}{#1}}
\newcommand{\DecValTok}[1]{\textcolor[rgb]{0.00,0.00,0.81}{#1}}
\newcommand{\DocumentationTok}[1]{\textcolor[rgb]{0.56,0.35,0.01}{\textbf{\textit{#1}}}}
\newcommand{\ErrorTok}[1]{\textcolor[rgb]{0.64,0.00,0.00}{\textbf{#1}}}
\newcommand{\ExtensionTok}[1]{#1}
\newcommand{\FloatTok}[1]{\textcolor[rgb]{0.00,0.00,0.81}{#1}}
\newcommand{\FunctionTok}[1]{\textcolor[rgb]{0.00,0.00,0.00}{#1}}
\newcommand{\ImportTok}[1]{#1}
\newcommand{\InformationTok}[1]{\textcolor[rgb]{0.56,0.35,0.01}{\textbf{\textit{#1}}}}
\newcommand{\KeywordTok}[1]{\textcolor[rgb]{0.13,0.29,0.53}{\textbf{#1}}}
\newcommand{\NormalTok}[1]{#1}
\newcommand{\OperatorTok}[1]{\textcolor[rgb]{0.81,0.36,0.00}{\textbf{#1}}}
\newcommand{\OtherTok}[1]{\textcolor[rgb]{0.56,0.35,0.01}{#1}}
\newcommand{\PreprocessorTok}[1]{\textcolor[rgb]{0.56,0.35,0.01}{\textit{#1}}}
\newcommand{\RegionMarkerTok}[1]{#1}
\newcommand{\SpecialCharTok}[1]{\textcolor[rgb]{0.00,0.00,0.00}{#1}}
\newcommand{\SpecialStringTok}[1]{\textcolor[rgb]{0.31,0.60,0.02}{#1}}
\newcommand{\StringTok}[1]{\textcolor[rgb]{0.31,0.60,0.02}{#1}}
\newcommand{\VariableTok}[1]{\textcolor[rgb]{0.00,0.00,0.00}{#1}}
\newcommand{\VerbatimStringTok}[1]{\textcolor[rgb]{0.31,0.60,0.02}{#1}}
\newcommand{\WarningTok}[1]{\textcolor[rgb]{0.56,0.35,0.01}{\textbf{\textit{#1}}}}
\usepackage{graphicx,grffile}
\makeatletter
\def\maxwidth{\ifdim\Gin@nat@width>\linewidth\linewidth\else\Gin@nat@width\fi}
\def\maxheight{\ifdim\Gin@nat@height>\textheight\textheight\else\Gin@nat@height\fi}
\makeatother
% Scale images if necessary, so that they will not overflow the page
% margins by default, and it is still possible to overwrite the defaults
% using explicit options in \includegraphics[width, height, ...]{}
\setkeys{Gin}{width=\maxwidth,height=\maxheight,keepaspectratio}
\setlength{\emergencystretch}{3em}  % prevent overfull lines
\providecommand{\tightlist}{%
  \setlength{\itemsep}{0pt}\setlength{\parskip}{0pt}}
\setcounter{secnumdepth}{-2}
% Redefines (sub)paragraphs to behave more like sections
\ifx\paragraph\undefined\else
  \let\oldparagraph\paragraph
  \renewcommand{\paragraph}[1]{\oldparagraph{#1}\mbox{}}
\fi
\ifx\subparagraph\undefined\else
  \let\oldsubparagraph\subparagraph
  \renewcommand{\subparagraph}[1]{\oldsubparagraph{#1}\mbox{}}
\fi

% set default figure placement to htbp
\makeatletter
\def\fps@figure{htbp}
\makeatother


\author{}
\date{\vspace{-2.5em}}

\begin{document}

u--- title: ``Team Project 5'' output: html\_notebook ---

\hypertarget{load-data}{%
\subsection{Load data}\label{load-data}}

\begin{Shaded}
\begin{Highlighting}[]
\CommentTok{# importing files}
\KeywordTok{require}\NormalTok{(readr)}
\end{Highlighting}
\end{Shaded}

\begin{verbatim}
## Loading required package: readr
\end{verbatim}

\begin{Shaded}
\begin{Highlighting}[]
\CommentTok{# data processing}
\KeywordTok{require}\NormalTok{(dplyr)}
\end{Highlighting}
\end{Shaded}

\begin{verbatim}
## Loading required package: dplyr
\end{verbatim}

\begin{verbatim}
## 
## Attaching package: 'dplyr'
\end{verbatim}

\begin{verbatim}
## The following objects are masked from 'package:stats':
## 
##     filter, lag
\end{verbatim}

\begin{verbatim}
## The following objects are masked from 'package:base':
## 
##     intersect, setdiff, setequal, union
\end{verbatim}

\begin{Shaded}
\begin{Highlighting}[]
\KeywordTok{require}\NormalTok{(tidyr)}
\end{Highlighting}
\end{Shaded}

\begin{verbatim}
## Loading required package: tidyr
\end{verbatim}

\begin{Shaded}
\begin{Highlighting}[]
\KeywordTok{require}\NormalTok{(lubridate) }\CommentTok{# for date processing}
\end{Highlighting}
\end{Shaded}

\begin{verbatim}
## Loading required package: lubridate
\end{verbatim}

\begin{verbatim}
## 
## Attaching package: 'lubridate'
\end{verbatim}

\begin{verbatim}
## The following objects are masked from 'package:base':
## 
##     date, intersect, setdiff, union
\end{verbatim}

\begin{Shaded}
\begin{Highlighting}[]
\CommentTok{# Map making}
\KeywordTok{library}\NormalTok{(picante)}
\end{Highlighting}
\end{Shaded}

\begin{verbatim}
## Warning: package 'picante' was built under R version 4.0.5
\end{verbatim}

\begin{verbatim}
## Loading required package: ape
\end{verbatim}

\begin{verbatim}
## Warning: package 'ape' was built under R version 4.0.5
\end{verbatim}

\begin{verbatim}
## Loading required package: vegan
\end{verbatim}

\begin{verbatim}
## Loading required package: permute
\end{verbatim}

\begin{verbatim}
## Loading required package: lattice
\end{verbatim}

\begin{verbatim}
## This is vegan 2.5-7
\end{verbatim}

\begin{verbatim}
## Loading required package: nlme
\end{verbatim}

\begin{verbatim}
## 
## Attaching package: 'nlme'
\end{verbatim}

\begin{verbatim}
## The following object is masked from 'package:dplyr':
## 
##     collapse
\end{verbatim}

\begin{Shaded}
\begin{Highlighting}[]
\KeywordTok{library}\NormalTok{(sf)}
\end{Highlighting}
\end{Shaded}

\begin{verbatim}
## Warning: package 'sf' was built under R version 4.0.5
\end{verbatim}

\begin{verbatim}
## Linking to GEOS 3.9.0, GDAL 3.2.1, PROJ 7.2.1
\end{verbatim}

\begin{Shaded}
\begin{Highlighting}[]
\KeywordTok{library}\NormalTok{(tmap)}
\end{Highlighting}
\end{Shaded}

\begin{verbatim}
## Warning: package 'tmap' was built under R version 4.0.5
\end{verbatim}

\begin{Shaded}
\begin{Highlighting}[]
\CommentTok{# Visualization}
\KeywordTok{library}\NormalTok{(ggplot2)}
\CommentTok{# library(ggtree)}

\CommentTok{# Qbio}
\CommentTok{## Phylogenies}
\KeywordTok{library}\NormalTok{(phyloregion)}
\end{Highlighting}
\end{Shaded}

\begin{verbatim}
## Warning: package 'phyloregion' was built under R version 4.0.5
\end{verbatim}

\begin{Shaded}
\begin{Highlighting}[]
\KeywordTok{library}\NormalTok{(ape)}
\KeywordTok{require}\NormalTok{(}\StringTok{'vegan'}\NormalTok{)}

\CommentTok{# Beta diversity}
\KeywordTok{library}\NormalTok{(pvclust)}
\end{Highlighting}
\end{Shaded}

\begin{verbatim}
## Warning: package 'pvclust' was built under R version 4.0.5
\end{verbatim}

\begin{Shaded}
\begin{Highlighting}[]
\NormalTok{mydata <-}\StringTok{ }\KeywordTok{read_csv}\NormalTok{(}\DataTypeTok{file =} \StringTok{"DeamDataInitial.csv"}\NormalTok{)}
\end{Highlighting}
\end{Shaded}

\begin{verbatim}
## 
## -- Column specification --------------------------------------------------------
## cols(
##   .default = col_character(),
##   id = col_double(),
##   collectionCode = col_logical(),
##   ownerInstitutionCode = col_logical(),
##   otherCatalogNumbers = col_double(),
##   taxonID = col_double(),
##   subgenus = col_logical(),
##   identificationReferences = col_logical(),
##   identificationRemarks = col_logical(),
##   taxonRemarks = col_logical(),
##   identificationQualifier = col_logical(),
##   year = col_double(),
##   month = col_double(),
##   day = col_double(),
##   startDayOfYear = col_double(),
##   endDayOfYear = col_double(),
##   fieldNumber = col_logical(),
##   informationWithheld = col_logical(),
##   dataGeneralizations = col_logical(),
##   dynamicProperties = col_logical(),
##   establishmentMeans = col_logical()
##   # ... with 19 more columns
## )
## i Use `spec()` for the full column specifications.
\end{verbatim}

\begin{verbatim}
## Warning: 329 parsing failures.
##  row                      col           expected                   actual                  file
## 1132 verbatimCoordinates      1/0/T/F/TRUE/FALSE ,                        'DeamDataInitial.csv'
## 1574 identificationRemarks    1/0/T/F/TRUE/FALSE Nomenclatural adjustment 'DeamDataInitial.csv'
## 1579 identificationReferences 1/0/T/F/TRUE/FALSE FNA                      'DeamDataInitial.csv'
## 1741 identificationReferences 1/0/T/F/TRUE/FALSE FNA                      'DeamDataInitial.csv'
## 1887 identificationReferences 1/0/T/F/TRUE/FALSE FNA                      'DeamDataInitial.csv'
## .... ........................ .................. ........................ .....................
## See problems(...) for more details.
\end{verbatim}

\hypertarget{keep-desired-columns}{%
\subsection{Keep desired columns}\label{keep-desired-columns}}

\begin{Shaded}
\begin{Highlighting}[]
\NormalTok{df1 <-}\StringTok{ }\KeywordTok{subset}\NormalTok{(mydata, }\DataTypeTok{select =} \KeywordTok{c}\NormalTok{(family, genus, specificEpithet, recordedBy, eventDate, country, stateProvince, county, decimalLatitude, decimalLongitude, geodeticDatum, coordinateUncertaintyInMeters))}
\end{Highlighting}
\end{Shaded}

\hypertarget{combine-genus-and-specificepithet-and-then-remove-original-columns}{%
\subsection{Combine genus and specificEpithet and then remove original
columns}\label{combine-genus-and-specificepithet-and-then-remove-original-columns}}

\begin{Shaded}
\begin{Highlighting}[]
\NormalTok{df1}\OperatorTok{$}\NormalTok{speciesName <-}\StringTok{ }\KeywordTok{paste}\NormalTok{(df1}\OperatorTok{$}\NormalTok{genus,df1}\OperatorTok{$}\NormalTok{specificEpithet)}
\NormalTok{df2 <-}\StringTok{ }\KeywordTok{subset}\NormalTok{(df1, }\DataTypeTok{select =} \KeywordTok{c}\NormalTok{(family, speciesName, recordedBy, eventDate, country, stateProvince, county, decimalLatitude, decimalLongitude))}
\end{Highlighting}
\end{Shaded}

\hypertarget{remove-data-with-no-gps-coordinates-or-dates}{%
\subsection{Remove data with no GPS coordinates or
dates}\label{remove-data-with-no-gps-coordinates-or-dates}}

\begin{Shaded}
\begin{Highlighting}[]
\NormalTok{df3 <-}\StringTok{ }\NormalTok{df2 }\OperatorTok\StringTok{ }\KeywordTok{drop_na}\NormalTok{()}
\end{Highlighting}
\end{Shaded}

\hypertarget{remove-data-that-are-not-identified-to-species-denoted-by-na-and-remove-hybrids-denoted-by-uxf3}{%
\subsection{Remove data that are not identified to species (denoted by
``NA'') and remove hybrids (denoted by
``√ó'')?}\label{remove-data-that-are-not-identified-to-species-denoted-by-na-and-remove-hybrids-denoted-by-uxf3}}

\begin{Shaded}
\begin{Highlighting}[]
\KeywordTok{library}\NormalTok{(tidyverse)}
\end{Highlighting}
\end{Shaded}

\begin{verbatim}
## -- Attaching packages --------------------------------------- tidyverse 1.3.0 --
\end{verbatim}

\begin{verbatim}
## v tibble  3.1.0     v stringr 1.4.0
## v purrr   0.3.4     v forcats 0.5.1
\end{verbatim}

\begin{verbatim}
## -- Conflicts ------------------------------------------ tidyverse_conflicts() --
## x lubridate::as.difftime() masks base::as.difftime()
## x nlme::collapse()         masks dplyr::collapse()
## x lubridate::date()        masks base::date()
## x dplyr::filter()          masks stats::filter()
## x lubridate::intersect()   masks base::intersect()
## x dplyr::lag()             masks stats::lag()
## x lubridate::setdiff()     masks base::setdiff()
## x lubridate::union()       masks base::union()
\end{verbatim}

\begin{Shaded}
\begin{Highlighting}[]
\NormalTok{df4 <-}\StringTok{ }\NormalTok{df3 }\OperatorTok\StringTok{ }\KeywordTok{filter_all}\NormalTok{(}\KeywordTok{all_vars}\NormalTok{(}\OperatorTok{!}\KeywordTok{grepl}\NormalTok{(}\StringTok{"NA"}\NormalTok{, .)))}
\NormalTok{df5 <-}\StringTok{ }\NormalTok{df4 }\OperatorTok\StringTok{ }\KeywordTok{filter_all}\NormalTok{(}\KeywordTok{all_vars}\NormalTok{(}\OperatorTok{!}\KeywordTok{grepl}\NormalTok{(}\StringTok{"ó"}\NormalTok{, .)))}
\end{Highlighting}
\end{Shaded}

\hypertarget{correct-county-name-spelling}{%
\subsection{Correct county name
spelling}\label{correct-county-name-spelling}}

\begin{Shaded}
\begin{Highlighting}[]
\CommentTok{## Upload county name list}
\NormalTok{ctydata <-}\StringTok{ }\KeywordTok{read.csv}\NormalTok{(}\StringTok{"ctydata.csv"}\NormalTok{, }\DataTypeTok{header =} \OtherTok{TRUE}\NormalTok{, }\DataTypeTok{fileEncoding=}\StringTok{"UTF-8-BOM"}\NormalTok{)}
\CommentTok{# Test for wrong county spellings}
\NormalTok{x <-}\StringTok{ }\NormalTok{df5}\OperatorTok{$}\NormalTok{county }\OperatorTok\StringTok{ }\NormalTok{ctydata}\OperatorTok{$}\NormalTok{county}
\NormalTok{y <-}\StringTok{ }\KeywordTok{data.frame}\NormalTok{(x)}
\NormalTok{z <-}\StringTok{ }\KeywordTok{data.frame}\NormalTok{(}\KeywordTok{which}\NormalTok{(}\KeywordTok{grepl}\NormalTok{(}\StringTok{"FALSE"}\NormalTok{, y}\OperatorTok{$}\NormalTok{x)))}
\CommentTok{## If z contains data fix}
\CommentTok{## Correct county name spelling}
\NormalTok{df5}\OperatorTok{$}\NormalTok{county[df5}\OperatorTok{$}\NormalTok{county }\OperatorTok{==}\StringTok{ "knox"}\NormalTok{] <-}\StringTok{ "Knox"}
\NormalTok{df5}\OperatorTok{$}\NormalTok{county[df5}\OperatorTok{$}\NormalTok{county }\OperatorTok{==}\StringTok{ "gibson"}\NormalTok{] <-}\StringTok{ "Gibson"}
\NormalTok{df5}\OperatorTok{$}\NormalTok{county[df5}\OperatorTok{$}\NormalTok{county }\OperatorTok{==}\StringTok{ "cass"}\NormalTok{] <-}\StringTok{ "Cass"}
\NormalTok{df5}\OperatorTok{$}\NormalTok{county[df5}\OperatorTok{$}\NormalTok{county }\OperatorTok{==}\StringTok{ "pike"}\NormalTok{] <-}\StringTok{ "Pike"}
\NormalTok{df5}\OperatorTok{$}\NormalTok{county[df5}\OperatorTok{$}\NormalTok{county }\OperatorTok{==}\StringTok{ "orange"}\NormalTok{] <-}\StringTok{ "Orange"}
\NormalTok{df5}\OperatorTok{$}\NormalTok{county[df5}\OperatorTok{$}\NormalTok{county }\OperatorTok{==}\StringTok{ "grant"}\NormalTok{] <-}\StringTok{ "Grant"}
\NormalTok{df5}\OperatorTok{$}\NormalTok{county[df5}\OperatorTok{$}\NormalTok{county }\OperatorTok{==}\StringTok{ "wayne"}\NormalTok{] <-}\StringTok{ "Wayne"}
\NormalTok{df5}\OperatorTok{$}\NormalTok{county[df5}\OperatorTok{$}\NormalTok{county }\OperatorTok{==}\StringTok{ "jasper"}\NormalTok{] <-}\StringTok{ "Jasper"}
\NormalTok{df5}\OperatorTok{$}\NormalTok{county[df5}\OperatorTok{$}\NormalTok{county }\OperatorTok{==}\StringTok{ "ELkhart"}\NormalTok{] <-}\StringTok{ "Elkhart"}
\NormalTok{df5}\OperatorTok{$}\NormalTok{county[df5}\OperatorTok{$}\NormalTok{county }\OperatorTok{==}\StringTok{ "crawford"}\NormalTok{] <-}\StringTok{ "Crawford"}
\CommentTok{# Test for wrong county spellings}
\NormalTok{x <-}\StringTok{ }\NormalTok{df5}\OperatorTok{$}\NormalTok{county }\OperatorTok\StringTok{ }\NormalTok{ctydata}\OperatorTok{$}\NormalTok{county}
\NormalTok{y <-}\StringTok{ }\KeywordTok{data.frame}\NormalTok{(x)}
\NormalTok{z <-}\StringTok{ }\KeywordTok{data.frame}\NormalTok{(}\KeywordTok{which}\NormalTok{(}\KeywordTok{grepl}\NormalTok{(}\StringTok{"FALSE"}\NormalTok{, y}\OperatorTok{$}\NormalTok{x)))}
\CommentTok{## If z contains data fix}
\end{Highlighting}
\end{Shaded}

\hypertarget{correct-indiana-name-spelling}{%
\subsection{Correct Indiana name
spelling}\label{correct-indiana-name-spelling}}

\begin{Shaded}
\begin{Highlighting}[]
\CommentTok{# Check for records not Indiana}
\KeywordTok{table}\NormalTok{(df5[,}\DecValTok{5}\NormalTok{])}
\end{Highlighting}
\end{Shaded}

\begin{verbatim}
## 
## United States 
##         47762
\end{verbatim}

\begin{Shaded}
\begin{Highlighting}[]
\CommentTok{# Correct indiana to Indiana}
\NormalTok{df5}\OperatorTok{$}\NormalTok{stateProvince[df5}\OperatorTok{$}\NormalTok{stateProvince }\OperatorTok{==}\StringTok{ "indiana"}\NormalTok{] <-}\StringTok{ "Indiana"}
\CommentTok{# Check again}
\KeywordTok{table}\NormalTok{(df5[,}\DecValTok{5}\NormalTok{])}
\end{Highlighting}
\end{Shaded}

\begin{verbatim}
## 
## United States 
##         47762
\end{verbatim}

\hypertarget{correct-species-names-against-official-list-at-httpsmidwestherbaria.orgportalchecklistschecklist.phpclid3510pid124dynclid0.-note-that-4-species-not-on-that-list-are-included-here-because-they-are-exotic-species-validly-documented-in-the-state.}{%
\subsection{\texorpdfstring{Correct species names against official list
at:
\url{https://midwestherbaria.org/portal/checklists/checklist.php?clid=3510\&pid=124\&dynclid=0}.
Note that 4 species not on that list are included here because they are
exotic species validly documented in the
state.}{Correct species names against official list at: https://midwestherbaria.org/portal/checklists/checklist.php?clid=3510\&pid=124\&dynclid=0. Note that 4 species not on that list are included here because they are exotic species validly documented in the state.}}\label{correct-species-names-against-official-list-at-httpsmidwestherbaria.orgportalchecklistschecklist.phpclid3510pid124dynclid0.-note-that-4-species-not-on-that-list-are-included-here-because-they-are-exotic-species-validly-documented-in-the-state.}}

\begin{Shaded}
\begin{Highlighting}[]
\CommentTok{## Upload county name list}
\NormalTok{indspecies <-}\StringTok{ }\KeywordTok{read.csv}\NormalTok{(}\StringTok{"indspecies.csv"}\NormalTok{, }\DataTypeTok{header =} \OtherTok{TRUE}\NormalTok{, }\DataTypeTok{fileEncoding=}\StringTok{"UTF-8"}\NormalTok{)}
\CommentTok{# Test for wrong species spellings}
\NormalTok{x <-}\StringTok{ }\NormalTok{df5}\OperatorTok{$}\NormalTok{speciesName }\OperatorTok\StringTok{ }\NormalTok{indspecies}\OperatorTok{$}\NormalTok{ScientificName}
\NormalTok{y <-}\StringTok{ }\KeywordTok{data.frame}\NormalTok{(x)}
\NormalTok{z <-}\StringTok{ }\KeywordTok{data.frame}\NormalTok{(}\KeywordTok{which}\NormalTok{(}\KeywordTok{grepl}\NormalTok{(}\StringTok{"FALSE"}\NormalTok{, y}\OperatorTok{$}\NormalTok{x)))}
\CommentTok{## Fix data}
\NormalTok{df5}\OperatorTok{$}\NormalTok{speciesName[df5}\OperatorTok{$}\NormalTok{speciesName }\OperatorTok{==}\StringTok{ "Andropogon gerardii"}\NormalTok{] <-}\StringTok{ "Andropogon gerardi"}
\NormalTok{df5}\OperatorTok{$}\NormalTok{speciesName[df5}\OperatorTok{$}\NormalTok{speciesName }\OperatorTok{==}\StringTok{ "Adicea deamii"}\NormalTok{] <-}\StringTok{ "Pilea pumila"}
\NormalTok{df5}\OperatorTok{$}\NormalTok{speciesName[df5}\OperatorTok{$}\NormalTok{speciesName }\OperatorTok{==}\StringTok{ "Baptisia lactea"}\NormalTok{] <-}\StringTok{ "Baptisia alba"}
\NormalTok{df5}\OperatorTok{$}\NormalTok{speciesName[df5}\OperatorTok{$}\NormalTok{speciesName }\OperatorTok{==}\StringTok{ "Lacinaria scariosa"}\NormalTok{] <-}\StringTok{ "Liatris scariosa"}
\NormalTok{df5}\OperatorTok{$}\NormalTok{speciesName[df5}\OperatorTok{$}\NormalTok{speciesName }\OperatorTok{==}\StringTok{ "Carex striatula"}\NormalTok{] <-}\StringTok{ "Carex laxiflora"}
\NormalTok{df5}\OperatorTok{$}\NormalTok{speciesName[df5}\OperatorTok{$}\NormalTok{speciesName }\OperatorTok{==}\StringTok{ "Viburnum pubescens"}\NormalTok{] <-}\StringTok{ "Viburnum dentatum"}
\NormalTok{df5}\OperatorTok{$}\NormalTok{speciesName[df5}\OperatorTok{$}\NormalTok{speciesName }\OperatorTok{==}\StringTok{ "Celtis pumila"}\NormalTok{] <-}\StringTok{ "Celtis occidentalis"}
\NormalTok{df5}\OperatorTok{$}\NormalTok{speciesName[df5}\OperatorTok{$}\NormalTok{speciesName }\OperatorTok{==}\StringTok{ "Vernonia altissima"}\NormalTok{] <-}\StringTok{ "Vernonia gigantea"}
\NormalTok{df5}\OperatorTok{$}\NormalTok{speciesName <-}\StringTok{ }\KeywordTok{as.character}\NormalTok{(df5}\OperatorTok{$}\NormalTok{speciesName)}
\NormalTok{df5}\OperatorTok{$}\NormalTok{speciesName[}\KeywordTok{grepl}\NormalTok{(}\StringTok{"Malus io"}\NormalTok{, df5}\OperatorTok{$}\NormalTok{speciesName)] <-}\StringTok{ "Malus ioensis"}
\NormalTok{df5}\OperatorTok{$}\NormalTok{speciesName <-}\StringTok{ }\KeywordTok{as.factor}\NormalTok{(df5}\OperatorTok{$}\NormalTok{speciesName)}
\NormalTok{df5}\OperatorTok{$}\NormalTok{speciesName[df5}\OperatorTok{$}\NormalTok{speciesName }\OperatorTok{==}\StringTok{ "Botrychium obliquum"}\NormalTok{] <-}\StringTok{ "Sceptridium dissectum"}
\NormalTok{df5}\OperatorTok{$}\NormalTok{speciesName[df5}\OperatorTok{$}\NormalTok{speciesName }\OperatorTok{==}\StringTok{ "Arabis viridis"}\NormalTok{] <-}\StringTok{ "Borodinia missouriensis"}
\NormalTok{df5}\OperatorTok{$}\NormalTok{speciesName <-}\StringTok{ }\KeywordTok{as.character}\NormalTok{(df5}\OperatorTok{$}\NormalTok{speciesName)}
\NormalTok{df5}\OperatorTok{$}\NormalTok{speciesName[}\KeywordTok{grepl}\NormalTok{(}\StringTok{"tes engelmannii"}\NormalTok{, df5}\OperatorTok{$}\NormalTok{speciesName)] <-}\StringTok{ "Isoetes engelmannii"}
\NormalTok{df5}\OperatorTok{$}\NormalTok{speciesName <-}\StringTok{ }\KeywordTok{as.factor}\NormalTok{(df5}\OperatorTok{$}\NormalTok{speciesName)}
\NormalTok{df5}\OperatorTok{$}\NormalTok{speciesName[df5}\OperatorTok{$}\NormalTok{speciesName }\OperatorTok{==}\StringTok{ "Juncus macer"}\NormalTok{] <-}\StringTok{ "Juncus anthelatus"}
\NormalTok{df5}\OperatorTok{$}\NormalTok{speciesName[df5}\OperatorTok{$}\NormalTok{speciesName }\OperatorTok{==}\StringTok{ "Solidago deamii"}\NormalTok{] <-}\StringTok{ "Solidago simplex"}
\NormalTok{df5}\OperatorTok{$}\NormalTok{speciesName[df5}\OperatorTok{$}\NormalTok{speciesName }\OperatorTok{==}\StringTok{ "Rosa deamii"}\NormalTok{] <-}\StringTok{ "Rosa carolina"}
\NormalTok{df5}\OperatorTok{$}\NormalTok{speciesName[df5}\OperatorTok{$}\NormalTok{speciesName }\OperatorTok{==}\StringTok{ "Panicum deamii"}\NormalTok{] <-}\StringTok{ "Dichanthelium ovale"}
\NormalTok{df5}\OperatorTok{$}\NormalTok{speciesName <-}\StringTok{ }\KeywordTok{as.character}\NormalTok{(df5}\OperatorTok{$}\NormalTok{speciesName)}
\NormalTok{df5}\OperatorTok{$}\NormalTok{speciesName[df5}\OperatorTok{$}\NormalTok{speciesName }\OperatorTok{==}\StringTok{ "Piptatherum pungens"}\NormalTok{] <-}\StringTok{ "Piptatheropsis pungens"}
\NormalTok{df5}\OperatorTok{$}\NormalTok{speciesName[df5}\OperatorTok{$}\NormalTok{speciesName }\OperatorTok{==}\StringTok{ "Helianthus rigidus"}\NormalTok{] <-}\StringTok{ "Helianthus pauciflorus"}
\NormalTok{df5}\OperatorTok{$}\NormalTok{speciesName[df5}\OperatorTok{$}\NormalTok{speciesName }\OperatorTok{==}\StringTok{ "Malus angustifolia"}\NormalTok{] <-}\StringTok{ "Malus ioensis"}
\CommentTok{# Remove remaining hybrids}
\NormalTok{df6 <-}\StringTok{ }\NormalTok{df5[}\OperatorTok{!}\NormalTok{df5}\OperatorTok{$}\NormalTok{speciesName }\OperatorTok{==}\StringTok{ "Lacinaria deamii"}\NormalTok{, ]}
\NormalTok{df7 <-}\StringTok{ }\NormalTok{df6[}\OperatorTok{!}\NormalTok{df6}\OperatorTok{$}\NormalTok{speciesName }\OperatorTok{==}\StringTok{ "Crataegus incaedua"}\NormalTok{, ]}
\CommentTok{# Check species again}
\NormalTok{x <-}\StringTok{ }\NormalTok{df7}\OperatorTok{$}\NormalTok{speciesName }\OperatorTok\StringTok{ }\NormalTok{indspecies}\OperatorTok{$}\NormalTok{ScientificName}
\NormalTok{y <-}\StringTok{ }\KeywordTok{data.frame}\NormalTok{(x)}
\NormalTok{z <-}\StringTok{ }\KeywordTok{data.frame}\NormalTok{(}\KeywordTok{which}\NormalTok{(}\KeywordTok{grepl}\NormalTok{(}\StringTok{"FALSE"}\NormalTok{, y}\OperatorTok{$}\NormalTok{x)))}
\end{Highlighting}
\end{Shaded}

\hypertarget{export-df7-to-csv-as-filtered-data}{%
\subsection{Export df7 to csv as filtered
data}\label{export-df7-to-csv-as-filtered-data}}

\begin{Shaded}
\begin{Highlighting}[]
\KeywordTok{write.csv}\NormalTok{(df7, }\StringTok{"Deam_Collection_Data_Filtered_3-27-2021.csv"}\NormalTok{)}
\end{Highlighting}
\end{Shaded}

\hypertarget{create-site-species-matrix-for-the-complete-dataset-treating-counties-as-sites}{%
\subsection{Create site-species matrix for the complete dataset,
treating counties as
sites}\label{create-site-species-matrix-for-the-complete-dataset-treating-counties-as-sites}}

\begin{Shaded}
\begin{Highlighting}[]
\CommentTok{# Subset data to only include species name and county}
\NormalTok{df8 <-}\StringTok{ }\KeywordTok{subset}\NormalTok{(df7, }\DataTypeTok{select =} \KeywordTok{c}\NormalTok{(county, speciesName))}
\CommentTok{# Sum number of occurrences of each species in each county}

\NormalTok{df9 <-}\StringTok{ }\NormalTok{df8 }\OperatorTok\StringTok{ }\KeywordTok{count}\NormalTok{(county, speciesName)}
\CommentTok{# Create site x species matrix by reshaping data}
\NormalTok{compdatasitesp <-}\StringTok{ }\KeywordTok{pivot_wider}\NormalTok{(df9, }\DataTypeTok{names_from =}\NormalTok{ speciesName, }\DataTypeTok{values_from =}\NormalTok{ n,)}
\CommentTok{# Replace NA's with 0}
\NormalTok{compdatasitesp[}\KeywordTok{is.na}\NormalTok{(compdatasitesp)] <-}\StringTok{ }\DecValTok{0}

\CommentTok{# convert back to data frame and name counties for sites. }
\NormalTok{compdatasitesp <-}\StringTok{ }\KeywordTok{as.data.frame}\NormalTok{(compdatasitesp)}
\KeywordTok{rownames}\NormalTok{(compdatasitesp) <-}\StringTok{ }\NormalTok{compdatasitesp[,}\DecValTok{1}\NormalTok{]}
\NormalTok{compdatasitesp <-}\StringTok{ }\NormalTok{compdatasitesp[,}\OperatorTok{-}\DecValTok{1}\NormalTok{]}
\end{Highlighting}
\end{Shaded}

\hypertarget{spatial-phylogenetic-analysis}{%
\subsection{SPATIAL PHYLOGENETIC
ANALYSIS}\label{spatial-phylogenetic-analysis}}

\hypertarget{start-spatial-phylogenetic-analysis-of-woody-plants}{%
\subsection{Start spatial phylogenetic analysis of woody
plants}\label{start-spatial-phylogenetic-analysis-of-woody-plants}}

\hypertarget{import-woody-tree-species-phylogeny-and-visualize}{%
\subsection{Import woody tree species phylogeny and
visualize}\label{import-woody-tree-species-phylogeny-and-visualize}}

\begin{Shaded}
\begin{Highlighting}[]
\CommentTok{# Import phlogenetic tree}
\CommentTok{# Data from: 10.1111/jbi.12171}
\NormalTok{woodytreesphylo <-}\StringTok{ }\KeywordTok{read.tree}\NormalTok{(}\DataTypeTok{file =} \StringTok{"Woody_Trees_Phylogeny.txt"}\NormalTok{)}
\CommentTok{# Plot phylogeny}
\KeywordTok{plot}\NormalTok{(woodytreesphylo, }\DataTypeTok{show.tip.label=}\OtherTok{FALSE}\NormalTok{)}
\end{Highlighting}
\end{Shaded}

\includegraphics{TeamProject5_files/figure-latex/unnamed-chunk-12-1.pdf}

\hypertarget{subset-deam-data-by-woody-tree-species-present-in-phylogeny}{%
\subsection{Subset Deam data by woody tree species present in
phylogeny}\label{subset-deam-data-by-woody-tree-species-present-in-phylogeny}}

\begin{Shaded}
\begin{Highlighting}[]
\CommentTok{# Replace "_" symbol with " " in phylogeny label data}
\NormalTok{woodytreesphylo}\OperatorTok{$}\NormalTok{tip.label <-}\StringTok{ }\KeywordTok{gsub}\NormalTok{(}\StringTok{"_"}\NormalTok{, }\StringTok{" "}\NormalTok{, woodytreesphylo}\OperatorTok{$}\NormalTok{tip.label, }\DataTypeTok{fixed=}\OtherTok{TRUE}\NormalTok{)}
\CommentTok{# Establish dataframe with tree species names}
\NormalTok{woodytreesp <-}\StringTok{ }\KeywordTok{data.frame}\NormalTok{(woodytreesphylo}\OperatorTok{$}\NormalTok{tip.label)}
\KeywordTok{colnames}\NormalTok{(woodytreesp)[}\DecValTok{1}\NormalTok{] <-}\StringTok{ "speciesName"}
\CommentTok{# Subset Deam data using species names in phylogeny}
\NormalTok{woodytreedeam <-}\StringTok{ }\NormalTok{df7 }\OperatorTok
\StringTok{      }\KeywordTok{filter}\NormalTok{(df7}\OperatorTok{$}\NormalTok{speciesName }\OperatorTok\StringTok{ }\NormalTok{woodytreesp}\OperatorTok{$}\NormalTok{speciesName)}
\end{Highlighting}
\end{Shaded}

\hypertarget{calculate-number-of-species-in-indiana-and-number-of-occurrences-for-each-species}{%
\subsection{Calculate number of species in Indiana and number of
occurrences for each
species}\label{calculate-number-of-species-in-indiana-and-number-of-occurrences-for-each-species}}

\begin{Shaded}
\begin{Highlighting}[]
\CommentTok{# Number of occurrences of each species}
\NormalTok{speciescount <-}\StringTok{ }\KeywordTok{setNames}\NormalTok{(}\KeywordTok{as.data.frame}\NormalTok{(}\KeywordTok{table}\NormalTok{(woodytreedeam}\OperatorTok{$}\NormalTok{speciesName)), }\KeywordTok{c}\NormalTok{(}\StringTok{"speciesName"}\NormalTok{, }\StringTok{"n"}\NormalTok{))}
\CommentTok{# Number of occurences per county}
\NormalTok{countycount <-}\StringTok{ }\KeywordTok{setNames}\NormalTok{(}\KeywordTok{as.data.frame}\NormalTok{(}\KeywordTok{table}\NormalTok{(woodytreedeam}\OperatorTok{$}\NormalTok{county)), }\KeywordTok{c}\NormalTok{(}\StringTok{"county"}\NormalTok{, }\StringTok{"n"}\NormalTok{))}
\NormalTok{countycount}\OperatorTok{$}\NormalTok{county <-}\StringTok{ }\KeywordTok{as.character}\NormalTok{(countycount}\OperatorTok{$}\NormalTok{county)}
\NormalTok{countycount}\OperatorTok{$}\NormalTok{county[countycount}\OperatorTok{$}\NormalTok{county }\OperatorTok{==}\StringTok{ "La Porte"}\NormalTok{] <-}\StringTok{ "LaPorte"}
\end{Highlighting}
\end{Shaded}

\hypertarget{create-site-species-matrix-for-woody-tree-species-in-indiana}{%
\section{Create site species matrix for woody tree species in
Indiana}\label{create-site-species-matrix-for-woody-tree-species-in-indiana}}

\begin{Shaded}
\begin{Highlighting}[]
\CommentTok{# Subset data to only include species name and county}
\NormalTok{woodytreedeamssm1 <-}\StringTok{ }\KeywordTok{subset}\NormalTok{(woodytreedeam, }\DataTypeTok{select =} \KeywordTok{c}\NormalTok{(county, speciesName))}
\NormalTok{woodytreedeamssm2 <-}\StringTok{ }\NormalTok{woodytreedeamssm1 }\OperatorTok\StringTok{ }\KeywordTok{count}\NormalTok{(county, speciesName)}
\CommentTok{# Create site x species matrix by reshaping data}
\NormalTok{woodytreedeamssm3 <-}\StringTok{ }\KeywordTok{pivot_wider}\NormalTok{(woodytreedeamssm2, }\DataTypeTok{names_from =}\NormalTok{ speciesName, }\DataTypeTok{values_from =}\NormalTok{ n,)}
\CommentTok{# Replace NA's with 0}
\NormalTok{woodytreedeamssm3[}\KeywordTok{is.na}\NormalTok{(woodytreedeamssm3)] <-}\StringTok{ }\DecValTok{0}
\CommentTok{# convert back to data frame and name counties for sites. }
\NormalTok{woodytreedeamssm3 <-}\StringTok{ }\KeywordTok{as.data.frame}\NormalTok{(woodytreedeamssm3)}
\KeywordTok{rownames}\NormalTok{(woodytreedeamssm3) <-}\StringTok{ }\NormalTok{woodytreedeamssm3[,}\DecValTok{1}\NormalTok{]}
\NormalTok{woodytreedeamssm3 <-}\StringTok{ }\NormalTok{woodytreedeamssm3[,}\OperatorTok{-}\DecValTok{1}\NormalTok{]}
\end{Highlighting}
\end{Shaded}

\hypertarget{calculate-faiths-phylogenetic-diversity-per-indiana-county}{%
\section{Calculate Faith's phylogenetic diversity per Indiana
county}\label{calculate-faiths-phylogenetic-diversity-per-indiana-county}}

\begin{Shaded}
\begin{Highlighting}[]
\CommentTok{# install and load package}
\CommentTok{#install.packages("picante")}

\CommentTok{# Calculate Faith's phylogenetic diversity}
\NormalTok{fpdwoody <-}\StringTok{ }\KeywordTok{pd}\NormalTok{(woodytreedeamssm3, woodytreesphylo, }\DataTypeTok{include.root=}\OtherTok{TRUE}\NormalTok{)}
\CommentTok{# Set row header to new column}
\NormalTok{fpdwoody}\OperatorTok{$}\NormalTok{county <-}\StringTok{ }\KeywordTok{rownames}\NormalTok{(fpdwoody)}
\CommentTok{# Change La Porte to "LaPorte"}
\NormalTok{fpdwoody}\OperatorTok{$}\NormalTok{county[fpdwoody}\OperatorTok{$}\NormalTok{county }\OperatorTok{==}\StringTok{ "La Porte"}\NormalTok{] <-}\StringTok{ "LaPorte"}
\end{Highlighting}
\end{Shaded}

\hypertarget{upload-indiana-county-maps}{%
\section{Upload Indiana county maps}\label{upload-indiana-county-maps}}

\begin{Shaded}
\begin{Highlighting}[]
\CommentTok{# upload shapefile with indiana counties}
\NormalTok{indcounties1 <-}\StringTok{ }\KeywordTok{st_read}\NormalTok{(}\StringTok{"Indiana_Counties.shp"}\NormalTok{)}
\end{Highlighting}
\end{Shaded}

\begin{verbatim}
## Reading layer `Indiana_Counties' from data source `C:\Users\Rich Hull\GitHub\teamproject5\Indiana_Counties.shp' using driver `ESRI Shapefile'
## Simple feature collection with 92 features and 10 fields
## Geometry type: POLYGON
## Dimension:     XY
## Bounding box:  xmin: -88.09789 ymin: 37.77174 xmax: -84.78458 ymax: 41.76059
## Geodetic CRS:  NAD83
\end{verbatim}

\begin{Shaded}
\begin{Highlighting}[]
\CommentTok{# subset indiana counties}
\NormalTok{indcounties2 <-}\StringTok{ }\KeywordTok{subset}\NormalTok{(indcounties1, }\DataTypeTok{select =} \KeywordTok{c}\NormalTok{(NAME, geometry))}
\CommentTok{# change crs to wgs84}
\NormalTok{indcounties3 <-}\StringTok{ }\KeywordTok{st_transform}\NormalTok{(indcounties2, }\DecValTok{4326}\NormalTok{)}
\CommentTok{# rename "NAME" column to "county"}
\KeywordTok{colnames}\NormalTok{(indcounties3)[}\DecValTok{1}\NormalTok{] <-}\StringTok{ "county"}
\CommentTok{# insert county centroid point}
\NormalTok{indcounties4 <-}\StringTok{ }\NormalTok{indcounties3}
\NormalTok{indcounties4}\OperatorTok{$}\NormalTok{centroid <-}\StringTok{ }\KeywordTok{st_centroid}\NormalTok{(indcounties4}\OperatorTok{$}\NormalTok{geometry)}
\NormalTok{indcounties4}\OperatorTok{$}\NormalTok{xy <-}\StringTok{ }\KeywordTok{st_coordinates}\NormalTok{(indcounties4}\OperatorTok{$}\NormalTok{centroid)}
\end{Highlighting}
\end{Shaded}

\hypertarget{plot-heat-map-of-faiths-phylogenetic-diversity-for-woody-trees-of-indiana-on-per-county-basis}{%
\section{Plot heat map of Faith's phylogenetic diversity for woody trees
of Indiana on per county
basis}\label{plot-heat-map-of-faiths-phylogenetic-diversity-for-woody-trees-of-indiana-on-per-county-basis}}

\begin{Shaded}
\begin{Highlighting}[]
\CommentTok{# Merge data}
\NormalTok{map <-}\StringTok{ }\KeywordTok{merge}\NormalTok{(indcounties3,}
\NormalTok{                     fpdwoody,}
                     \DataTypeTok{by.x =} \StringTok{"county"}\NormalTok{,}
                     \DataTypeTok{by.y =} \StringTok{"county"}\NormalTok{)}
\CommentTok{# Plot}
\KeywordTok{tm_shape}\NormalTok{(map, }\DataTypeTok{projection=}\StringTok{"+init=epsg:4326"}\NormalTok{) }\OperatorTok{+}
\StringTok{  }\KeywordTok{tm_polygons}\NormalTok{(}\StringTok{"PD"}\NormalTok{, }\DataTypeTok{border.col =} \StringTok{"grey30"}\NormalTok{, }\DataTypeTok{title=}\StringTok{""}\NormalTok{) }\OperatorTok{+}
\KeywordTok{tm_layout}\NormalTok{(}\DataTypeTok{title=}\StringTok{"Faith's Phylogenetic Diversity"}\NormalTok{, }
          \DataTypeTok{title.size =} \DecValTok{1}\NormalTok{, }
          \DataTypeTok{title.position =} \KeywordTok{c}\NormalTok{(}\StringTok{"center"}\NormalTok{, }\StringTok{"top"}\NormalTok{), }
          \DataTypeTok{legend.text.size=}\NormalTok{.}\DecValTok{6}\NormalTok{, }\DataTypeTok{bg.color =} \StringTok{"light blue"}\NormalTok{, }\DataTypeTok{inner.margins=}\KeywordTok{c}\NormalTok{(.}\DecValTok{1}\NormalTok{,.}\DecValTok{1}\NormalTok{,.}\DecValTok{2}\NormalTok{,.}\DecValTok{4}\NormalTok{))}
\end{Highlighting}
\end{Shaded}

\begin{verbatim}
## Warning in CPL_crs_from_input(x): GDAL Message 1: +init=epsg:XXXX syntax is
## deprecated. It might return a CRS with a non-EPSG compliant axis order.
\end{verbatim}

\includegraphics{TeamProject5_files/figure-latex/unnamed-chunk-18-1.pdf}

\begin{Shaded}
\begin{Highlighting}[]
\KeywordTok{tm_legend}\NormalTok{(}\DataTypeTok{position =} \KeywordTok{c}\NormalTok{(}\StringTok{"right"}\NormalTok{, }\StringTok{"bottom"}\NormalTok{), }
    \DataTypeTok{frame =} \OtherTok{TRUE}\NormalTok{,}
    \DataTypeTok{bg.color=}\StringTok{"lightblue"}\NormalTok{)}
\end{Highlighting}
\end{Shaded}

\begin{verbatim}
## $tm_layout
## $tm_layout$legend.position
## [1] "right"  "bottom"
## 
## $tm_layout$legend.frame
## [1] TRUE
## 
## $tm_layout$legend.bg.color
## [1] "lightblue"
## 
## $tm_layout$style
## [1] NA
## 
## 
## attr(,"class")
## [1] "tm"
\end{verbatim}

\hypertarget{calculate-and-plot-alpha-diversity-of-woody-tree-species-in-indiana}{%
\subsection{Calculate and plot alpha diversity of woody tree species in
Indiana}\label{calculate-and-plot-alpha-diversity-of-woody-tree-species-in-indiana}}

\begin{Shaded}
\begin{Highlighting}[]
\CommentTok{# define species richness function with site species matrix as input}
\NormalTok{S.obs <-}\StringTok{ }\ControlFlowTok{function}\NormalTok{(}\DataTypeTok{x =} \StringTok{""}\NormalTok{)\{}
  \KeywordTok{rowSums}\NormalTok{(x }\OperatorTok{>}\StringTok{ }\DecValTok{0}\NormalTok{) }\OperatorTok{*}\StringTok{ }\DecValTok{1}
\NormalTok{\}}
\CommentTok{# calculate site species richness (county) using site species matrix of Indiana woody tree species}
\NormalTok{obssprichwt <-}\StringTok{ }\KeywordTok{data.frame}\NormalTok{(}\KeywordTok{S.obs}\NormalTok{(woodytreedeamssm3))}
\KeywordTok{colnames}\NormalTok{(obssprichwt)[}\DecValTok{1}\NormalTok{] <-}\StringTok{ "speciesRichness"}
\CommentTok{# Set row header to new column}
\NormalTok{obssprichwt}\OperatorTok{$}\NormalTok{county <-}\StringTok{ }\KeywordTok{rownames}\NormalTok{(obssprichwt)}
\CommentTok{# Change La Porte to "LaPorte"}
\NormalTok{obssprichwt}\OperatorTok{$}\NormalTok{county[obssprichwt}\OperatorTok{$}\NormalTok{county }\OperatorTok{==}\StringTok{ "La Porte"}\NormalTok{] <-}\StringTok{ "LaPorte"}
\CommentTok{# Plot species richness as heat map per county}
\CommentTok{# Merge data}
\NormalTok{map1 <-}\StringTok{ }\KeywordTok{merge}\NormalTok{(indcounties3,}
\NormalTok{                     obssprichwt,}
                     \DataTypeTok{by.x =} \StringTok{"county"}\NormalTok{,}
                     \DataTypeTok{by.y =} \StringTok{"county"}\NormalTok{)}
\CommentTok{# Plot}
\KeywordTok{tm_shape}\NormalTok{(map1, }\DataTypeTok{projection=}\StringTok{"+init=epsg:4326"}\NormalTok{) }\OperatorTok{+}
\StringTok{  }\KeywordTok{tm_polygons}\NormalTok{(}\StringTok{"speciesRichness"}\NormalTok{, }\DataTypeTok{border.col =} \StringTok{"grey30"}\NormalTok{, }\DataTypeTok{title=}\StringTok{""}\NormalTok{) }\OperatorTok{+}
\KeywordTok{tm_layout}\NormalTok{(}\DataTypeTok{title=}\StringTok{"Observed Species Richness"}\NormalTok{, }
          \DataTypeTok{title.size =} \DecValTok{1}\NormalTok{, }
          \DataTypeTok{title.position =} \KeywordTok{c}\NormalTok{(}\StringTok{"center"}\NormalTok{, }\StringTok{"top"}\NormalTok{), }
          \DataTypeTok{legend.text.size=}\NormalTok{.}\DecValTok{6}\NormalTok{, }\DataTypeTok{bg.color =} \StringTok{"light blue"}\NormalTok{, }\DataTypeTok{inner.margins=}\KeywordTok{c}\NormalTok{(.}\DecValTok{1}\NormalTok{,.}\DecValTok{1}\NormalTok{,.}\DecValTok{2}\NormalTok{,.}\DecValTok{4}\NormalTok{))}
\end{Highlighting}
\end{Shaded}

\includegraphics{TeamProject5_files/figure-latex/unnamed-chunk-19-1.pdf}

\begin{Shaded}
\begin{Highlighting}[]
\KeywordTok{tm_legend}\NormalTok{(}\DataTypeTok{position =} \KeywordTok{c}\NormalTok{(}\StringTok{"right"}\NormalTok{, }\StringTok{"bottom"}\NormalTok{), }
    \DataTypeTok{frame =} \OtherTok{TRUE}\NormalTok{,}
    \DataTypeTok{bg.color=}\StringTok{"lightblue"}\NormalTok{)}
\end{Highlighting}
\end{Shaded}

\begin{verbatim}
## $tm_layout
## $tm_layout$legend.position
## [1] "right"  "bottom"
## 
## $tm_layout$legend.frame
## [1] TRUE
## 
## $tm_layout$legend.bg.color
## [1] "lightblue"
## 
## $tm_layout$style
## [1] NA
## 
## 
## attr(,"class")
## [1] "tm"
\end{verbatim}

\hypertarget{test-for-significance-of-woody-tree-alpha-diversity-in-southern-section-of-state}{%
\subsection{Test for significance of woody tree alpha diversity in
southern section of
state}\label{test-for-significance-of-woody-tree-alpha-diversity-in-southern-section-of-state}}

\begin{Shaded}
\begin{Highlighting}[]
\CommentTok{# import county data}
\NormalTok{ctydatalat <-}\StringTok{ }\KeywordTok{read.csv}\NormalTok{(}\StringTok{"IndianaCountiesSN.csv"}\NormalTok{, }\DataTypeTok{header =} \OtherTok{TRUE}\NormalTok{, }\DataTypeTok{fileEncoding=}\StringTok{"UTF-8-BOM"}\NormalTok{)}
\CommentTok{# merge with species richness data}
\NormalTok{ctyspeciesrich <-}\StringTok{ }\KeywordTok{merge}\NormalTok{(ctydatalat,}
\NormalTok{                     obssprichwt,}
                     \DataTypeTok{by.x =} \StringTok{"county"}\NormalTok{,}
                     \DataTypeTok{by.y =} \StringTok{"county"}\NormalTok{)}
\CommentTok{# perform t-test}
\KeywordTok{t.test}\NormalTok{(speciesRichness }\OperatorTok{~}\StringTok{ }\NormalTok{latitude, }\DataTypeTok{alternative =} \StringTok{"less"}\NormalTok{, }\DataTypeTok{data =}\NormalTok{ ctyspeciesrich)}
\end{Highlighting}
\end{Shaded}

\begin{verbatim}
## 
##  Welch Two Sample t-test
## 
## data:  speciesRichness by latitude
## t = -2.0465, df = 26.253, p-value = 0.02542
## alternative hypothesis: true difference in means is less than 0
## 95 percent confidence interval:
##       -Inf -1.453648
## sample estimates:
## mean in group N mean in group S 
##        31.38889        40.10000
\end{verbatim}

\hypertarget{test-for-correlation-between-alpha-diversity-and-phylogenetic-diversity}{%
\subsection{Test for correlation between alpha diversity and
phylogenetic
diversity}\label{test-for-correlation-between-alpha-diversity-and-phylogenetic-diversity}}

\begin{Shaded}
\begin{Highlighting}[]
\CommentTok{# summarize data}
\KeywordTok{summary}\NormalTok{(fpdwoody)}
\end{Highlighting}
\end{Shaded}

\begin{verbatim}
##        PD               SR           county         
##  Min.   : 652.3   Min.   : 6.00   Length:92         
##  1st Qu.:1819.9   1st Qu.:23.75   Class :character  
##  Median :2218.9   Median :33.00   Mode  :character  
##  Mean   :2269.2   Mean   :33.28                     
##  3rd Qu.:2725.8   3rd Qu.:40.00                     
##  Max.   :3751.7   Max.   :67.00
\end{verbatim}

\begin{Shaded}
\begin{Highlighting}[]
\CommentTok{# check for normality}
\KeywordTok{hist}\NormalTok{(fpdwoody}\OperatorTok{$}\NormalTok{PD)}
\end{Highlighting}
\end{Shaded}

\includegraphics{TeamProject5_files/figure-latex/unnamed-chunk-21-1.pdf}

\begin{Shaded}
\begin{Highlighting}[]
\KeywordTok{hist}\NormalTok{(fpdwoody}\OperatorTok{$}\NormalTok{SR)}
\end{Highlighting}
\end{Shaded}

\includegraphics{TeamProject5_files/figure-latex/unnamed-chunk-21-2.pdf}

\begin{Shaded}
\begin{Highlighting}[]
\CommentTok{# take log of data}
\NormalTok{fpdwoody}\OperatorTok{$}\NormalTok{logpd <-}\StringTok{ }\KeywordTok{log}\NormalTok{(fpdwoody}\OperatorTok{$}\NormalTok{PD)}
\NormalTok{fpdwoody}\OperatorTok{$}\NormalTok{logsr <-}\StringTok{ }\KeywordTok{log}\NormalTok{(fpdwoody}\OperatorTok{$}\NormalTok{SR)}
\CommentTok{# perform linear regression}
\NormalTok{pd.sr.lm <-}\StringTok{ }\KeywordTok{lm}\NormalTok{(logsr }\OperatorTok{~}\StringTok{ }\NormalTok{logpd, }\DataTypeTok{data =}\NormalTok{ fpdwoody)}
\KeywordTok{summary}\NormalTok{(pd.sr.lm)}
\end{Highlighting}
\end{Shaded}

\begin{verbatim}
## 
## Call:
## lm(formula = logsr ~ logpd, data = fpdwoody)
## 
## Residuals:
##       Min        1Q    Median        3Q       Max 
## -0.209559 -0.074841  0.001887  0.060110  0.268357 
## 
## Coefficients:
##             Estimate Std. Error t value Pr(>|t|)    
## (Intercept) -7.53382    0.20841  -36.15   <2e-16 ***
## logpd        1.42380    0.02716   52.43   <2e-16 ***
## ---
## Signif. codes:  0 '***' 0.001 '**' 0.01 '*' 0.05 '.' 0.1 ' ' 1
## 
## Residual standard error: 0.09697 on 90 degrees of freedom
## Multiple R-squared:  0.9683, Adjusted R-squared:  0.9679 
## F-statistic:  2749 on 1 and 90 DF,  p-value: < 2.2e-16
\end{verbatim}

\begin{Shaded}
\begin{Highlighting}[]
\CommentTok{# visualize linear regression}
\NormalTok{alphaphylo.graph <-}\StringTok{ }\KeywordTok{ggplot}\NormalTok{(fpdwoody, }\KeywordTok{aes}\NormalTok{(}\DataTypeTok{x=}\NormalTok{logpd, }\DataTypeTok{y=}\NormalTok{logsr))}\OperatorTok{+}
\StringTok{                     }\KeywordTok{geom_point}\NormalTok{() }\OperatorTok{+}\StringTok{ }\KeywordTok{geom_smooth}\NormalTok{(}\DataTypeTok{method=}\StringTok{"lm"}\NormalTok{, }\DataTypeTok{col=}\StringTok{"black"}\NormalTok{) }\OperatorTok{+}\StringTok{ }\KeywordTok{labs}\NormalTok{(}\DataTypeTok{title =} \StringTok{"Species Richness vs. Phlyogenetic Diversity"}\NormalTok{,}
      \DataTypeTok{x =} \StringTok{"Faith's Phylogenetic Diversity (log)"}\NormalTok{,}
      \DataTypeTok{y =} \StringTok{"Species Richness (log)"}\NormalTok{)}
\NormalTok{alphaphylo.graph}
\end{Highlighting}
\end{Shaded}

\begin{verbatim}
## `geom_smooth()` using formula 'y ~ x'
\end{verbatim}

\includegraphics{TeamProject5_files/figure-latex/unnamed-chunk-21-3.pdf}

\hypertarget{test-for-unequal-scaling-between-alpha-and-phylogenetic-diversity-across-the-state}{%
\subsection{Test for unequal scaling between alpha and phylogenetic
diversity across the
state}\label{test-for-unequal-scaling-between-alpha-and-phylogenetic-diversity-across-the-state}}

\begin{Shaded}
\begin{Highlighting}[]
\CommentTok{# Calculate PD/SR ratio}
\NormalTok{fpdwoody}\OperatorTok{$}\NormalTok{pdsrratio <-}\StringTok{ }\NormalTok{fpdwoody}\OperatorTok{$}\NormalTok{PD }\OperatorTok{/}\StringTok{ }\NormalTok{fpdwoody}\OperatorTok{$}\NormalTok{SR}
\CommentTok{# import county data}
\NormalTok{ctydatascaling <-}\StringTok{ }\KeywordTok{read.csv}\NormalTok{(}\StringTok{"IndianaCountiesScaling.csv"}\NormalTok{, }\DataTypeTok{header =} \OtherTok{TRUE}\NormalTok{, }\DataTypeTok{fileEncoding=}\StringTok{"UTF-8-BOM"}\NormalTok{)}
\CommentTok{# merge with species richness data}
\NormalTok{ctyspeciesscale <-}\StringTok{ }\KeywordTok{merge}\NormalTok{(ctydatascaling,}
\NormalTok{                     fpdwoody,}
                     \DataTypeTok{by.x =} \StringTok{"county"}\NormalTok{,}
                     \DataTypeTok{by.y =} \StringTok{"county"}\NormalTok{)}
\NormalTok{ctyspeciesscale}\OperatorTok{$}\NormalTok{colors <-}\StringTok{ }\KeywordTok{as.factor}\NormalTok{(ctyspeciesscale}\OperatorTok{$}\NormalTok{latitude)}
\CommentTok{# run anova}
\NormalTok{anovapdsrratio <-}\StringTok{ }\KeywordTok{aov}\NormalTok{(pdsrratio }\OperatorTok{~}\StringTok{ }\NormalTok{latitude, }\DataTypeTok{data =}\NormalTok{ ctyspeciesscale)}
\KeywordTok{summary}\NormalTok{(anovapdsrratio)}
\end{Highlighting}
\end{Shaded}

\begin{verbatim}
##             Df Sum Sq Mean Sq F value   Pr(>F)    
## latitude     2   4122    2061   11.98 2.46e-05 ***
## Residuals   89  15307     172                     
## ---
## Signif. codes:  0 '***' 0.001 '**' 0.01 '*' 0.05 '.' 0.1 ' ' 1
\end{verbatim}

\begin{Shaded}
\begin{Highlighting}[]
\CommentTok{# visualize data via boxplot}
\KeywordTok{boxplot}\NormalTok{(pdsrratio }\OperatorTok{~}\StringTok{ }\NormalTok{latitude, }\DataTypeTok{data =}\NormalTok{ ctyspeciesscale, }\DataTypeTok{main =} \StringTok{"PD:AD Ratio across Three Regions"}\NormalTok{,}
   \DataTypeTok{xlab =} \StringTok{"Indiana Regions"}\NormalTok{, }\DataTypeTok{ylab =} \StringTok{"PD:AD"}\NormalTok{)}
\end{Highlighting}
\end{Shaded}

\includegraphics{TeamProject5_files/figure-latex/unnamed-chunk-22-1.pdf}

\begin{Shaded}
\begin{Highlighting}[]
\CommentTok{# Plot of data}
\NormalTok{alphaphylo.graph <-}\StringTok{ }\KeywordTok{ggplot}\NormalTok{(ctyspeciesscale, }\KeywordTok{aes}\NormalTok{(}\DataTypeTok{x=}\NormalTok{logpd, }\DataTypeTok{y=}\NormalTok{logsr,}\DataTypeTok{col=}\NormalTok{colors))}\OperatorTok{+}
\StringTok{                     }\KeywordTok{geom_point}\NormalTok{() }\OperatorTok{+}\StringTok{ }\KeywordTok{geom_smooth}\NormalTok{(}\DataTypeTok{method=}\StringTok{"lm"}\NormalTok{, }\DataTypeTok{col=}\StringTok{"black"}\NormalTok{) }\OperatorTok{+}\StringTok{ }\KeywordTok{labs}\NormalTok{(}\DataTypeTok{title =} \StringTok{"Species Richness vs. Phlyogenetic Diversity"}\NormalTok{,}
      \DataTypeTok{x =} \StringTok{"Faith's Phylogenetic Diversity (log)"}\NormalTok{,}
      \DataTypeTok{y =} \StringTok{"Species Richness (log)"}\NormalTok{)}
\NormalTok{alphaphylo.graph}
\end{Highlighting}
\end{Shaded}

\begin{verbatim}
## `geom_smooth()` using formula 'y ~ x'
\end{verbatim}

\includegraphics{TeamProject5_files/figure-latex/unnamed-chunk-22-2.pdf}

\hypertarget{calculate-and-plot-beta-diversity-of-woody-tree-species-in-indiana}{%
\subsection{Calculate and plot beta diversity of woody tree species in
Indiana}\label{calculate-and-plot-beta-diversity-of-woody-tree-species-in-indiana}}

\begin{Shaded}
\begin{Highlighting}[]
\CommentTok{# create matrix with counties as columns}
\NormalTok{transspsite <-}\StringTok{ }\KeywordTok{pivot_wider}\NormalTok{(woodytreedeamssm2, }\DataTypeTok{names_from =}\NormalTok{ county, }\DataTypeTok{values_from =}\NormalTok{ n,)}
\CommentTok{# Replace NA's with 0}
\NormalTok{transspsite[}\KeywordTok{is.na}\NormalTok{(transspsite)] <-}\StringTok{ }\DecValTok{0}
\CommentTok{# convert back to data frame and name counties for sites. }
\NormalTok{transspsite <-}\StringTok{ }\KeywordTok{as.data.frame}\NormalTok{(transspsite)}
\KeywordTok{rownames}\NormalTok{(transspsite) <-}\StringTok{ }\NormalTok{transspsite[,}\DecValTok{1}\NormalTok{]}
\NormalTok{transspsite <-}\StringTok{ }\NormalTok{transspsite[,}\OperatorTok{-}\DecValTok{1}\NormalTok{]}
\CommentTok{# perform cluster analysis}
\NormalTok{woodytree.ward <-}\StringTok{ }\KeywordTok{pvclust}\NormalTok{(transspsite, }\DataTypeTok{method.hclust =} \StringTok{"ward.D2"}\NormalTok{, }\DataTypeTok{iseed =} \DecValTok{65}\NormalTok{, }\DataTypeTok{nboot =} \DecValTok{10000}\NormalTok{)}
\CommentTok{# plot cluster}
\KeywordTok{par}\NormalTok{(}\DataTypeTok{mar =} \KeywordTok{c}\NormalTok{(}\DecValTok{1}\NormalTok{, }\DecValTok{5}\NormalTok{, }\DecValTok{2}\NormalTok{, }\DecValTok{2}\NormalTok{) }\OperatorTok{+}\StringTok{ }\FloatTok{0.1}\NormalTok{)}
\KeywordTok{plot}\NormalTok{(woodytree.ward, }\DataTypeTok{main =} \StringTok{"Woody Tree Species in Indiana: Ward's Clustering"}\NormalTok{, }\DataTypeTok{ylab =} \StringTok{"Squared Bray-Curtis Distance"}\NormalTok{, }\DataTypeTok{cex =} \FloatTok{0.6}\NormalTok{, }\DataTypeTok{col.pv =} \KeywordTok{c}\NormalTok{(}\DataTypeTok{au =} \DecValTok{3}\NormalTok{, }\DataTypeTok{bp =} \OtherTok{FALSE}\NormalTok{, }\DataTypeTok{edge =} \OtherTok{FALSE}\NormalTok{))}
\end{Highlighting}
\end{Shaded}

\hypertarget{test-for-ecoregion-clustering}{%
\section{Test for ecoregion
clustering}\label{test-for-ecoregion-clustering}}

\begin{Shaded}
\begin{Highlighting}[]
\CommentTok{# upload county ecoregion dataframe}
\NormalTok{ctydataecoreg <-}\StringTok{ }\KeywordTok{read.csv}\NormalTok{(}\StringTok{"IndianaCountiesEcoregionsIV.csv"}\NormalTok{, }\DataTypeTok{header =} \OtherTok{FALSE}\NormalTok{, }\DataTypeTok{fileEncoding=}\StringTok{"UTF-8-BOM"}\NormalTok{)}
\KeywordTok{rownames}\NormalTok{(ctydataecoreg) <-}\StringTok{ }\NormalTok{ctydataecoreg[,}\DecValTok{1}\NormalTok{]}
\NormalTok{ctydataecoreg <-}\StringTok{ }\NormalTok{ctydataecoreg[,}\OperatorTok{-}\DecValTok{1}\NormalTok{]}
\KeywordTok{as.factor}\NormalTok{(ctydataecoreg)}
\end{Highlighting}
\end{Shaded}

\begin{verbatim}
##  [1] Clayey High Lime Till Plains        Clayey High Lime Till Plains       
##  [3] Loamy High Lime Till Plains         Illinois/Indiana Prairies          
##  [5] Clayey High Lime Till Plains        Loamy High Lime Till Plains        
##  [7] Knobs-Norman Upland                 Loamy High Lime Till Plains        
##  [9] Loamy High Lime Till Plains         Pre-Wisconsinan Drift Plains       
## [11] Glaciated Wabash Lowlands           Loamy High Lime Till Plains        
## [13] Crawford-Mammoth Cave Uplands       Glaciated Wabash Lowlands          
## [15] Outer Bluegrass                     Loamy High Lime Till Plains        
## [17] Clayey High Lime Till Plains        Clayey High Lime Till Plains       
## [19] Crawford-Mammoth Cave Uplands       Battle Creek/Elkhart Outwash Plain 
## [21] Whitewater Interlobate Area         Mitchell Plain                     
## [23] Loamy High Lime Till Plains         Outer Bluegrass                    
## [25] Middle Tippecanoe Plains            Wabash-Ohio Bottomlands            
## [27] Clayey High Lime Till Plains        Glaciated Wabash Lowlands          
## [29] Loamy High Lime Till Plains         Loamy High Lime Till Plains        
## [31] Mitchell Plain                      Loamy High Lime Till Plains        
## [33] Whitewater Interlobate Area         Loamy High Lime Till Plains        
## [35] Clayey High Lime Till Plains        Pre-Wisconsinan Drift Plains       
## [37] Sand Area                           Clayey High Lime Till Plains       
## [39] Pre-Wisconsinan Drift Plains        Pre-Wisconsinan Drift Plains       
## [41] Loamy High Lime Till Plains         Glaciated Wabash Lowlands          
## [43] Northern Indiana Lake Country       Battle Creek/Elkhart Outwash Plain 
## [45] Valparaiso-Wheaton Morainal Complex Battle Creek/Elkhart Outwash Plain 
## [47] Mitchell Plain                      Loamy High Lime Till Plains        
## [49] Loamy High Lime Till Plains         Battle Creek/Elkhart Outwash Plain 
## [51] Crawford-Mammoth Cave Uplands       Clayey High Lime Till Plains       
## [53] Knobs-Norman Upland                 Loamy High Lime Till Plains        
## [55] Knobs-Norman Upland                 Sand Area                          
## [57] Outer Bluegrass                     Crawford-Mammoth Cave Uplands      
## [59] Crawford-Mammoth Cave Uplands       Loamy High Lime Till Plains        
## [61] Crawford-Mammoth Cave Uplands       Wabash-Ohio Bottomlands            
## [63] Northern Indiana Lake Country       Valparaiso-Wheaton Morainal Complex
## [65] Wabash-Ohio Bottomlands             Middle Tippecanoe Plains           
## [67] Loamy High Lime Till Plains         Clayey High Lime Till Plains       
## [69] Pre-Wisconsinan Drift Plains        Loamy High Lime Till Plains        
## [71] Pre-Wisconsinan Drift Plains        Loamy High Lime Till Plains        
## [73] Wabash-Ohio Bottomlands             Battle Creek/Elkhart Outwash Plain 
## [75] Sand Area                           Northern Indiana Lake Country      
## [77] Glaciated Wabash Lowlands           Outer Bluegrass                    
## [79] Loamy High Lime Till Plains         Loamy High Lime Till Plains        
## [81] Loamy High Lime Till Plains         Wabash-Ohio Bottomlands            
## [83] Glaciated Wabash Lowlands           Glaciated Wabash Lowlands          
## [85] Clayey High Lime Till Plains        Illinois/Indiana Prairies          
## [87] Wabash-Ohio Bottomlands             Mitchell Plain                     
## [89] Whitewater Interlobate Area         Clayey High Lime Till Plains       
## [91] Sand Area                           Clayey High Lime Till Plains       
## 16 Levels: Battle Creek/Elkhart Outwash Plain ... Whitewater Interlobate Area
\end{verbatim}

\begin{Shaded}
\begin{Highlighting}[]
\CommentTok{# run PERMANOVA with adonis function}
\KeywordTok{adonis}\NormalTok{(woodytreedeamssm3 }\OperatorTok{~}\StringTok{ }\NormalTok{ctydataecoreg, }\DataTypeTok{method =} \StringTok{"bray"}\NormalTok{, }\DataTypeTok{permutations =} \DecValTok{999}\NormalTok{)}
\end{Highlighting}
\end{Shaded}

\begin{verbatim}
## 
## Call:
## adonis(formula = woodytreedeamssm3 ~ ctydataecoreg, permutations = 999,      method = "bray") 
## 
## Permutation: free
## Number of permutations: 999
## 
## Terms added sequentially (first to last)
## 
##               Df SumsOfSqs MeanSqs F.Model      R2 Pr(>F)    
## ctydataecoreg 15    6.9402 0.46268  2.2732 0.30971  0.001 ***
## Residuals     76   15.4684 0.20353         0.69029           
## Total         91   22.4086                 1.00000           
## ---
## Signif. codes:  0 '***' 0.001 '**' 0.01 '*' 0.05 '.' 0.1 ' ' 1
\end{verbatim}

\begin{Shaded}
\begin{Highlighting}[]
\CommentTok{# upload county ecoregion dataframe}
\NormalTok{ctydataecoregiii <-}\StringTok{ }\KeywordTok{read.csv}\NormalTok{(}\StringTok{"IndianaCountiesEcoregionsIII.csv"}\NormalTok{, }\DataTypeTok{header =} \OtherTok{FALSE}\NormalTok{, }\DataTypeTok{fileEncoding=}\StringTok{"UTF-8-BOM"}\NormalTok{)}
\KeywordTok{rownames}\NormalTok{(ctydataecoregiii) <-}\StringTok{ }\NormalTok{ctydataecoregiii[,}\DecValTok{1}\NormalTok{]}
\NormalTok{ctydataecoregiii <-}\StringTok{ }\NormalTok{ctydataecoregiii[,}\OperatorTok{-}\DecValTok{1}\NormalTok{]}
\KeywordTok{as.factor}\NormalTok{(ctydataecoregiii)}
\end{Highlighting}
\end{Shaded}

\begin{verbatim}
##  [1] Eastern Corn Belt Plains                       
##  [2] Eastern Corn Belt Plains                       
##  [3] Eastern Corn Belt Plains                       
##  [4] Central Corn Belt Plains                       
##  [5] Eastern Corn Belt Plains                       
##  [6] Eastern Corn Belt Plains                       
##  [7] Interior Plateau                               
##  [8] Eastern Corn Belt Plains                       
##  [9] Southern Michigan/Northern Indiana Drift Plains
## [10] Eastern Corn Belt Plains                       
## [11] Interior River Valley and Hills                
## [12] Eastern Corn Belt Plains                       
## [13] Interior Plateau                               
## [14] Interior River Valley and Hills                
## [15] Interior Plateau                               
## [16] Eastern Corn Belt Plains                       
## [17] Eastern Corn Belt Plains                       
## [18] Eastern Corn Belt Plains                       
## [19] Interior Plateau                               
## [20] Southern Michigan/Northern Indiana Drift Plains
## [21] Eastern Corn Belt Plains                       
## [22] Interior Plateau                               
## [23] Eastern Corn Belt Plains                       
## [24] Eastern Corn Belt Plains                       
## [25] Southern Michigan/Northern Indiana Drift Plains
## [26] Interior River Valley and Hills                
## [27] Eastern Corn Belt Plains                       
## [28] Interior River Valley and Hills                
## [29] Eastern Corn Belt Plains                       
## [30] Eastern Corn Belt Plains                       
## [31] Interior Plateau                               
## [32] Eastern Corn Belt Plains                       
## [33] Eastern Corn Belt Plains                       
## [34] Eastern Corn Belt Plains                       
## [35] Eastern Corn Belt Plains                       
## [36] Eastern Corn Belt Plains                       
## [37] Central Corn Belt Plains                       
## [38] Eastern Corn Belt Plains                       
## [39] Eastern Corn Belt Plains                       
## [40] Eastern Corn Belt Plains                       
## [41] Eastern Corn Belt Plains                       
## [42] Interior River Valley and Hills                
## [43] Southern Michigan/Northern Indiana Drift Plains
## [44] Southern Michigan/Northern Indiana Drift Plains
## [45] Central Corn Belt Plains                       
## [46] Southern Michigan/Northern Indiana Drift Plains
## [47] Interior Plateau                               
## [48] Eastern Corn Belt Plains                       
## [49] Eastern Corn Belt Plains                       
## [50] Southern Michigan/Northern Indiana Drift Plains
## [51] Interior Plateau                               
## [52] Eastern Corn Belt Plains                       
## [53] Interior Plateau                               
## [54] Eastern Corn Belt Plains                       
## [55] Eastern Corn Belt Plains                       
## [56] Central Corn Belt Plains                       
## [57] Southern Michigan/Northern Indiana Drift Plains
## [58] Interior Plateau                               
## [59] Interior Plateau                               
## [60] Interior Plateau                               
## [61] Eastern Corn Belt Plains                       
## [62] Interior Plateau                               
## [63] Interior River Valley and Hills                
## [64] Central Corn Belt Plains                       
## [65] Interior River Valley and Hills                
## [66] Central Corn Belt Plains                       
## [67] Eastern Corn Belt Plains                       
## [68] Eastern Corn Belt Plains                       
## [69] Eastern Corn Belt Plains                       
## [70] Eastern Corn Belt Plains                       
## [71] Eastern Corn Belt Plains                       
## [72] Eastern Corn Belt Plains                       
## [73] Interior River Valley and Hills                
## [74] Southern Michigan/Northern Indiana Drift Plains
## [75] Central Corn Belt Plains                       
## [76] Southern Michigan/Northern Indiana Drift Plains
## [77] Interior River Valley and Hills                
## [78] Interior Plateau                               
## [79] Eastern Corn Belt Plains                       
## [80] Eastern Corn Belt Plains                       
## [81] Eastern Corn Belt Plains                       
## [82] Interior River Valley and Hills                
## [83] Interior River Valley and Hills                
## [84] Interior River Valley and Hills                
## [85] Eastern Corn Belt Plains                       
## [86] Central Corn Belt Plains                       
## [87] Interior River Valley and Hills                
## [88] Interior Plateau                               
## [89] Eastern Corn Belt Plains                       
## [90] Eastern Corn Belt Plains                       
## [91] Central Corn Belt Plains                       
## [92] Eastern Corn Belt Plains                       
## 5 Levels: Central Corn Belt Plains ... Southern Michigan/Northern Indiana Drift Plains
\end{verbatim}

\begin{Shaded}
\begin{Highlighting}[]
\CommentTok{# run PERMANOVA with adonis function}
\KeywordTok{adonis}\NormalTok{(woodytreedeamssm3 }\OperatorTok{~}\StringTok{ }\NormalTok{ctydataecoregiii, }\DataTypeTok{method =} \StringTok{"bray"}\NormalTok{, }\DataTypeTok{permutations =} \DecValTok{999}\NormalTok{)}
\end{Highlighting}
\end{Shaded}

\begin{verbatim}
## 
## Call:
## adonis(formula = woodytreedeamssm3 ~ ctydataecoregiii, permutations = 999,      method = "bray") 
## 
## Permutation: free
## Number of permutations: 999
## 
## Terms added sequentially (first to last)
## 
##                  Df SumsOfSqs MeanSqs F.Model      R2 Pr(>F)    
## ctydataecoregiii  4     3.138 0.78451  3.5418 0.14004  0.001 ***
## Residuals        87    19.271 0.22150         0.85996           
## Total            91    22.409                 1.00000           
## ---
## Signif. codes:  0 '***' 0.001 '**' 0.01 '*' 0.05 '.' 0.1 ' ' 1
\end{verbatim}

\hypertarget{create-heat-map-of-occurrences-of-woody-tree-species-in-indiana}{%
\section{Create heat map of occurrences of woody tree species in
Indiana}\label{create-heat-map-of-occurrences-of-woody-tree-species-in-indiana}}

\begin{Shaded}
\begin{Highlighting}[]
\CommentTok{# Merge data}
\NormalTok{map3 <-}\StringTok{ }\KeywordTok{merge}\NormalTok{(indcounties3,}
\NormalTok{                     countycount,}
                     \DataTypeTok{by.x =} \StringTok{"county"}\NormalTok{,}
                     \DataTypeTok{by.y =} \StringTok{"county"}\NormalTok{)}
\CommentTok{# Plot}
\KeywordTok{tm_shape}\NormalTok{(map3, }\DataTypeTok{projection=}\StringTok{"+init=epsg:4326"}\NormalTok{) }\OperatorTok{+}
\StringTok{  }\KeywordTok{tm_polygons}\NormalTok{(}\StringTok{"n"}\NormalTok{, }\DataTypeTok{border.col =} \StringTok{"grey30"}\NormalTok{, }\DataTypeTok{title=}\StringTok{""}\NormalTok{) }\OperatorTok{+}
\KeywordTok{tm_layout}\NormalTok{(}\DataTypeTok{title=}\StringTok{"Number of Occurrences of Woody Tree Species"}\NormalTok{, }
          \DataTypeTok{title.size =} \DecValTok{1}\NormalTok{, }
          \DataTypeTok{title.position =} \KeywordTok{c}\NormalTok{(}\StringTok{"center"}\NormalTok{, }\StringTok{"top"}\NormalTok{), }
          \DataTypeTok{legend.text.size=}\NormalTok{.}\DecValTok{6}\NormalTok{, }\DataTypeTok{bg.color =} \StringTok{"light blue"}\NormalTok{, }\DataTypeTok{inner.margins=}\KeywordTok{c}\NormalTok{(.}\DecValTok{1}\NormalTok{,.}\DecValTok{1}\NormalTok{,.}\DecValTok{2}\NormalTok{,.}\DecValTok{4}\NormalTok{))}
\end{Highlighting}
\end{Shaded}

\includegraphics{TeamProject5_files/figure-latex/unnamed-chunk-25-1.pdf}

\begin{Shaded}
\begin{Highlighting}[]
\KeywordTok{tm_legend}\NormalTok{(}\DataTypeTok{position =} \KeywordTok{c}\NormalTok{(}\StringTok{"right"}\NormalTok{, }\StringTok{"bottom"}\NormalTok{), }
    \DataTypeTok{frame =} \OtherTok{TRUE}\NormalTok{,}
    \DataTypeTok{bg.color=}\StringTok{"lightblue"}\NormalTok{)}
\end{Highlighting}
\end{Shaded}

\begin{verbatim}
## $tm_layout
## $tm_layout$legend.position
## [1] "right"  "bottom"
## 
## $tm_layout$legend.frame
## [1] TRUE
## 
## $tm_layout$legend.bg.color
## [1] "lightblue"
## 
## $tm_layout$style
## [1] NA
## 
## 
## attr(,"class")
## [1] "tm"
\end{verbatim}

\hypertarget{census-data-associations-with-species}{%
\section{Census Data Associations with
Species}\label{census-data-associations-with-species}}

\begin{Shaded}
\begin{Highlighting}[]
\NormalTok{S.obs <-}\StringTok{ }\ControlFlowTok{function}\NormalTok{(}\DataTypeTok{x=}\StringTok{''}\NormalTok{)\{}
  \KeywordTok{rowSums}\NormalTok{(x}\OperatorTok{>}\DecValTok{0}\NormalTok{)}\OperatorTok{*}\DecValTok{1}
\NormalTok{\}}
\end{Highlighting}
\end{Shaded}

What Census should we use?

\begin{Shaded}
\begin{Highlighting}[]
\NormalTok{a <-}\StringTok{ }\KeywordTok{parse_date_time}\NormalTok{(}\KeywordTok{as.vector}\NormalTok{(df7}\OperatorTok{$}\NormalTok{eventDate),}\DataTypeTok{orders =} \KeywordTok{c}\NormalTok{(}\StringTok{"mdy"}\NormalTok{,}\StringTok{"ymd"}\NormalTok{))}
\end{Highlighting}
\end{Shaded}

\begin{verbatim}
## Warning: 24 failed to parse.
\end{verbatim}

\begin{Shaded}
\begin{Highlighting}[]
\CommentTok{# Items with invalid Dates:}
\CommentTok{# df7[which(is.na(a)),]}
\KeywordTok{mean}\NormalTok{(a,}\DataTypeTok{na.rm=}\OtherTok{TRUE}\NormalTok{)}
\end{Highlighting}
\end{Shaded}

\begin{verbatim}
## [1] "1920-12-23 06:11:27 UTC"
\end{verbatim}

\begin{Shaded}
\begin{Highlighting}[]
\KeywordTok{median}\NormalTok{(a,}\DataTypeTok{na.rm=}\OtherTok{TRUE}\NormalTok{)}
\end{Highlighting}
\end{Shaded}

\begin{verbatim}
## [1] "1919-09-23 UTC"
\end{verbatim}

\begin{Shaded}
\begin{Highlighting}[]
\CommentTok{# US population Census 1920}
\NormalTok{USCensus <-}\StringTok{ }\KeywordTok{as.data.frame}\NormalTok{(}\KeywordTok{read.csv}\NormalTok{(}\StringTok{"data/IndianaCensus.csv"}\NormalTok{,}\DataTypeTok{header=}\OtherTok{TRUE}\NormalTok{))}
\KeywordTok{rownames}\NormalTok{(USCensus) <-}\StringTok{ }\NormalTok{USCensus[,}\DecValTok{1}\NormalTok{]}
\NormalTok{USCensus <-}\StringTok{ }\NormalTok{USCensus[,}\OperatorTok{-}\DecValTok{1}\NormalTok{]}
\CommentTok{# Census of Agriculture of 1920}
\NormalTok{AgCensus <-}\StringTok{ }\KeywordTok{as.data.frame}\NormalTok{(}\KeywordTok{read.csv}\NormalTok{(}\StringTok{"data/IndianaAgCensus1920.csv"}\NormalTok{,}\DataTypeTok{header=}\OtherTok{TRUE}\NormalTok{))}
\KeywordTok{rownames}\NormalTok{(AgCensus) <-}\StringTok{ }\NormalTok{AgCensus[,}\DecValTok{1}\NormalTok{]}
\NormalTok{AgCensus <-}\StringTok{ }\NormalTok{AgCensus[,}\OperatorTok{-}\DecValTok{1}\NormalTok{]}
\CommentTok{# Glaciation}
\NormalTok{Glacier <-}\StringTok{ }\KeywordTok{as.data.frame}\NormalTok{(}\KeywordTok{read.csv}\NormalTok{(}\StringTok{"data/Glaciation.csv"}\NormalTok{,}\DataTypeTok{header=}\OtherTok{TRUE}\NormalTok{))}
\CommentTok{#Glacier$Glaciation<- as.factor(Glacier$Glaciation)}
\CommentTok{# Ecoregions}
\NormalTok{EcoIII <-}\StringTok{ }\KeywordTok{as.data.frame}\NormalTok{(}\KeywordTok{read.csv}\NormalTok{(}\StringTok{"IndianaCountiesEcoregionsIII.csv"}\NormalTok{,}\DataTypeTok{header=}\OtherTok{FALSE}\NormalTok{,}\DataTypeTok{col.names =} \KeywordTok{c}\NormalTok{(}\StringTok{"county"}\NormalTok{,}\StringTok{"EcoIII"}\NormalTok{)))}
\CommentTok{#EcoIII$EcoIII <- as.factor(EcoIII$EcoIII)}
\NormalTok{EcoIV <-}\StringTok{ }\KeywordTok{as.data.frame}\NormalTok{(}\KeywordTok{read.csv}\NormalTok{(}\StringTok{"IndianaCountiesEcoregionsIV.csv"}\NormalTok{,}\DataTypeTok{header=}\OtherTok{FALSE}\NormalTok{,}\DataTypeTok{col.names =} \KeywordTok{c}\NormalTok{(}\StringTok{"county"}\NormalTok{,}\StringTok{"EcoIV"}\NormalTok{)))}
\CommentTok{#EcoIV$EcoIV <- as.factor(EcoIV$EcoIV)}
\end{Highlighting}
\end{Shaded}

\begin{Shaded}
\begin{Highlighting}[]
\NormalTok{popData <-}\StringTok{ }\KeywordTok{cbind}\NormalTok{(USCensus[}\OperatorTok{-}\DecValTok{1}\NormalTok{,],AgCensus[}\OperatorTok{-}\DecValTok{1}\NormalTok{,],EcoIII[}\OperatorTok{-}\DecValTok{1}\NormalTok{],Glacier[}\OperatorTok{-}\DecValTok{1}\NormalTok{],EcoIV[}\OperatorTok{-}\DecValTok{1}\NormalTok{],ctyspeciesscale}\OperatorTok{$}\NormalTok{latitude)}
\NormalTok{samplingEffort <-}\StringTok{ }\KeywordTok{as.data.frame}\NormalTok{(}\KeywordTok{rowSums}\NormalTok{(compdatasitesp))}
\KeywordTok{colnames}\NormalTok{(samplingEffort) <-}\StringTok{ }\KeywordTok{c}\NormalTok{(}\StringTok{"sampEffort"}\NormalTok{)}
\NormalTok{popData <-}\KeywordTok{cbind}\NormalTok{(popData,samplingEffort)}

\NormalTok{popData}\OperatorTok{$}\NormalTok{pcImproved1920 <-}\StringTok{ }\NormalTok{popData}\OperatorTok{$}\NormalTok{ImprovedFarmAcres1920}\OperatorTok{/}\NormalTok{popData}\OperatorTok{$}\NormalTok{LandArea1920}
\NormalTok{popData}\OperatorTok{$}\NormalTok{popDensity1920 <-}\StringTok{ }\NormalTok{popData}\OperatorTok{$}\NormalTok{X1920}\OperatorTok{/}\NormalTok{popData}\OperatorTok{$}\NormalTok{LandArea1920}


\NormalTok{woodytreedeamssm3.db <-}\StringTok{ }\KeywordTok{vegdist}\NormalTok{(woodytreedeamssm3,}\DataTypeTok{method=}\StringTok{'bray'}\NormalTok{,}\DataTypeTok{binary =} \OtherTok{TRUE}\NormalTok{,}\DataTypeTok{upper=}\OtherTok{TRUE}\NormalTok{,}\DataTypeTok{diag=}\OtherTok{TRUE}\NormalTok{)}

\NormalTok{woodytreedeamssm3.dbrda <-}\StringTok{ }\KeywordTok{dbrda}\NormalTok{(woodytreedeamssm3.db }\OperatorTok{~}\NormalTok{.,}\KeywordTok{as.data.frame}\NormalTok{(popData))}
\KeywordTok{ordiplot}\NormalTok{(woodytreedeamssm3.dbrda)}
\end{Highlighting}
\end{Shaded}

\includegraphics{TeamProject5_files/figure-latex/unnamed-chunk-29-1.pdf}
Cool, from this it is clear that the census population levels are
minimally correlated, while the land area of counties and improved acres
are more correlated.

\begin{Shaded}
\begin{Highlighting}[]
\NormalTok{woodytreedeamssm3.dbrda.mod0 <-}\KeywordTok{dbrda}\NormalTok{(woodytreedeamssm3.db}\OperatorTok{~}\DecValTok{1}\NormalTok{,}\KeywordTok{as.data.frame}\NormalTok{(popData))}
\CommentTok{# ordiplot(woodytreedeamssm3.dbrda.mod0)}
\end{Highlighting}
\end{Shaded}

\begin{Shaded}
\begin{Highlighting}[]
\NormalTok{woodytreedeamssm3.dbrda.mod1 <-}\StringTok{ }\KeywordTok{dbrda}\NormalTok{(woodytreedeamssm3.db}\OperatorTok{~}\NormalTok{., }\KeywordTok{as.data.frame}\NormalTok{(popData))}
\CommentTok{# ordiplot(woodytreedeamssm3.dbrda.mod1)}
\end{Highlighting}
\end{Shaded}

\begin{Shaded}
\begin{Highlighting}[]
\NormalTok{woodytreedeamssm3.dbrda <-}\StringTok{ }\KeywordTok{ordiR2step}\NormalTok{(woodytreedeamssm3.dbrda.mod0,woodytreedeamssm3.dbrda.mod1,}\DataTypeTok{perm.max=}\DecValTok{600}\NormalTok{)}
\end{Highlighting}
\end{Shaded}

\begin{verbatim}
## Step: R2.adj= 0 
## Call: woodytreedeamssm3.db ~ 1 
##  
##                                R2.adjusted
## <All variables>               2.704393e-01
## + EcoIV                       2.121690e-01
## + `ctyspeciesscale$latitude`  1.496147e-01
## + EcoIII                      1.328686e-01
## + sampEffort                  8.613621e-02
## + Glaciation                  4.896001e-02
## + pcImproved1920              2.455358e-02
## + LandArea1920                1.344601e-02
## + ImprovedFarmAcres1920       1.107929e-02
## + X1910                       1.469237e-04
## + X1900                       7.384027e-05
## <none>                        0.000000e+00
## + popDensity1920             -4.247067e-04
## + X1920                      -5.000502e-04
## + X1930                      -1.276819e-03
## + X1940                      -1.385791e-03
## + X1950                      -1.706708e-03
## + X1980                      -1.773088e-03
## + X1990                      -1.799462e-03
## + X2000                      -1.809320e-03
## + X1970                      -1.820925e-03
## + X2010                      -1.923783e-03
## + X1960                      -2.040614e-03
## 
##         Df    AIC      F Pr(>F)   
## + EcoIV 15 253.11 2.6338  0.002 **
## ---
## Signif. codes:  0 '***' 0.001 '**' 0.01 '*' 0.05 '.' 0.1 ' ' 1
## 
## Step: R2.adj= 0.212169 
## Call: woodytreedeamssm3.db ~ EcoIV 
##  
##                              R2.adjusted
## <All variables>                0.2704393
## + sampEffort                   0.2521396
## + `ctyspeciesscale$latitude`   0.2439664
## + EcoIII                       0.2202194
## + LandArea1920                 0.2179541
## + pcImproved1920               0.2168880
## + Glaciation                   0.2166639
## + ImprovedFarmAcres1920        0.2164552
## + popDensity1920               0.2144771
## + X1930                        0.2135119
## + X1920                        0.2135019
## + X1940                        0.2134368
## + X1910                        0.2132456
## + X1950                        0.2131270
## + X1960                        0.2126860
## + X1970                        0.2125884
## + X1980                        0.2123188
## + X1900                        0.2123107
## <none>                         0.2121690
## + X1990                        0.2120607
## + X2000                        0.2117218
## + X2010                        0.2112766
## 
##              Df   AIC      F Pr(>F)   
## + sampEffort  1 249.1 5.0619  0.002 **
## ---
## Signif. codes:  0 '***' 0.001 '**' 0.01 '*' 0.05 '.' 0.1 ' ' 1
## 
## Step: R2.adj= 0.2521396 
## Call: woodytreedeamssm3.db ~ EcoIV + sampEffort 
##  
##                              R2.adjusted
## + `ctyspeciesscale$latitude`   0.2792725
## <All variables>                0.2704393
## + EcoIII                       0.2584725
## + Glaciation                   0.2584129
## + pcImproved1920               0.2559731
## + ImprovedFarmAcres1920        0.2543502
## + X1920                        0.2540457
## + X1930                        0.2540226
## + X1940                        0.2539492
## + popDensity1920               0.2538165
## + X1910                        0.2537197
## + X1950                        0.2536528
## + LandArea1920                 0.2532996
## + X1960                        0.2532514
## + X1970                        0.2531599
## + X1980                        0.2528729
## + X1900                        0.2527266
## + X1990                        0.2525783
## + X2000                        0.2521707
## <none>                         0.2521396
## + X2010                        0.2516470
\end{verbatim}

\begin{Shaded}
\begin{Highlighting}[]
\NormalTok{woodytreedeamssm3.dbrda}\OperatorTok{$}\NormalTok{anova}
\end{Highlighting}
\end{Shaded}

\begin{verbatim}
##                  R2.adj Df    AIC      F Pr(>F)   
## + EcoIV         0.21217 15 253.11 2.6338  0.002 **
## + sampEffort    0.25214  1 249.10 5.0619  0.002 **
## <All variables> 0.27044                           
## ---
## Signif. codes:  0 '***' 0.001 '**' 0.01 '*' 0.05 '.' 0.1 ' ' 1
\end{verbatim}

\begin{Shaded}
\begin{Highlighting}[]
\KeywordTok{ordiplot}\NormalTok{(woodytreedeamssm3.dbrda)}
\end{Highlighting}
\end{Shaded}

\includegraphics{TeamProject5_files/figure-latex/unnamed-chunk-32-1.pdf}

\begin{Shaded}
\begin{Highlighting}[]
\KeywordTok{scores}\NormalTok{(woodytreedeamssm3.dbrda,}\DataTypeTok{display=}\StringTok{"bp"}\NormalTok{)}
\end{Highlighting}
\end{Shaded}

\begin{verbatim}
##                                                dbRDA1      dbRDA2
## EcoIVClayey High Lime Till Plains        -0.073673005 -0.24141639
## EcoIVCrawford-Mammoth Cave Uplands        0.190718800  0.23687969
## EcoIVGlaciated Wabash Lowlands           -0.004824015  0.16296218
## EcoIVIllinois/Indiana Prairies            0.121656252 -0.17357749
## EcoIVKnobs-Norman Upland                  0.096380018  0.22967507
## EcoIVLoamy High Lime Till Plains          0.533951222 -0.27616555
## EcoIVMiddle Tippecanoe Plains            -0.147465141 -0.13716061
## EcoIVMitchell Plain                      -0.032371151  0.38972282
## EcoIVNorthern Indiana Lake Country       -0.247057059 -0.02291336
## EcoIVOuter Bluegrass                     -0.023983369 -0.05278488
## EcoIVPre-Wisconsinan Drift Plains        -0.017121079  0.29999584
## EcoIVSand Area                           -0.357919947 -0.21877629
## EcoIVValparaiso-Wheaton Morainal Complex -0.294440810 -0.07852503
## EcoIVWabash-Ohio Bottomlands             -0.065990098  0.53099514
## EcoIVWhitewater Interlobate Area          0.191406390 -0.31195259
## sampEffort                               -0.855506680  0.22843624
## attr(,"const")
## [1] 6.270969
\end{verbatim}

OOF only 4\% of variation explained by these the agricultural
information we inputted, which probably isn't that bad given that we are
talking about species diversity across an entire state. Let's see what
happens with the information given by the spatial correlations.

\begin{Shaded}
\begin{Highlighting}[]
\NormalTok{rs <-}\StringTok{ }\KeywordTok{rowSums}\NormalTok{(woodytreedeamssm3)}\OperatorTok{/}\KeywordTok{sum}\NormalTok{(woodytreedeamssm3)}
\NormalTok{woodytreedeamssm3.pcnmw <-}\StringTok{ }\KeywordTok{pcnm}\NormalTok{(}\KeywordTok{dist}\NormalTok{(indcounties4}\OperatorTok{$}\NormalTok{xy),}\DataTypeTok{w=}\NormalTok{rs,}\DataTypeTok{dist.ret =} \OtherTok{TRUE}\NormalTok{)}

\NormalTok{woodytreedeamssm3.space <-}\KeywordTok{as.data.frame}\NormalTok{(}\KeywordTok{scores}\NormalTok{(woodytreedeamssm3.pcnmw))}
\NormalTok{woodytreedeamssm3.pcnm.mod0 <-}\KeywordTok{dbrda}\NormalTok{(woodytreedeamssm3.db }\OperatorTok{~}\DecValTok{1}\NormalTok{,woodytreedeamssm3.space)}
\NormalTok{woodytreedeamssm3.pcnm.mod1 <-}\KeywordTok{dbrda}\NormalTok{(woodytreedeamssm3.db }\OperatorTok{~}\NormalTok{.,woodytreedeamssm3.space)}
\NormalTok{step.pcnm <-}\StringTok{ }\KeywordTok{ordiR2step}\NormalTok{(woodytreedeamssm3.pcnm.mod0, woodytreedeamssm3.pcnm.mod1,}\DataTypeTok{perm.max=}\DecValTok{500}\NormalTok{)}
\end{Highlighting}
\end{Shaded}

\begin{verbatim}
## Step: R2.adj= 0 
## Call: woodytreedeamssm3.db ~ 1 
##  
##                   R2.adjusted
## <All variables>  1.575081e-02
## + PCNM37         1.460080e-02
## + PCNM44         1.427025e-02
## + PCNM34         1.382675e-02
## + PCNM52         1.378564e-02
## + PCNM43         1.141820e-02
## + PCNM56         9.621112e-03
## + PCNM39         8.535197e-03
## + PCNM40         8.477722e-03
## + PCNM45         6.684049e-03
## + PCNM51         6.526788e-03
## + PCNM38         5.868984e-03
## + PCNM46         5.242842e-03
## + PCNM29         5.220529e-03
## + PCNM55         4.400612e-03
## + PCNM53         4.206932e-03
## + PCNM42         3.864874e-03
## + PCNM35         3.734065e-03
## + PCNM13         3.647412e-03
## + PCNM26         3.618724e-03
## + PCNM58         3.463491e-03
## + PCNM24         2.993461e-03
## + PCNM36         2.823206e-03
## + PCNM7          2.768001e-03
## + PCNM33         2.135353e-03
## + PCNM4          1.964552e-03
## + PCNM54         1.769164e-03
## + PCNM18         1.101828e-03
## + PCNM41         6.778790e-04
## + PCNM25         6.682848e-04
## + PCNM5          6.299269e-04
## + PCNM14         4.853354e-04
## + PCNM32         4.295830e-04
## + PCNM47         3.851783e-04
## + PCNM19         2.305874e-04
## <none>           0.000000e+00
## + PCNM8         -4.775077e-06
## + PCNM49        -3.629386e-04
## + PCNM28        -6.818206e-04
## + PCNM30        -8.440009e-04
## + PCNM61        -1.051129e-03
## + PCNM15        -2.082247e-03
## + PCNM21        -2.555759e-03
## + PCNM57        -2.643770e-03
## + PCNM48        -2.742177e-03
## + PCNM2         -2.765693e-03
## + PCNM17        -2.849725e-03
## + PCNM23        -3.129529e-03
## + PCNM12        -3.232465e-03
## + PCNM9         -3.436056e-03
## + PCNM10        -3.489548e-03
## + PCNM59        -3.547983e-03
## + PCNM31        -3.601738e-03
## + PCNM11        -4.109548e-03
## + PCNM16        -4.454623e-03
## + PCNM3         -4.711601e-03
## + PCNM50        -5.083159e-03
## + PCNM1         -5.391661e-03
## + PCNM6         -5.447930e-03
## + PCNM27        -5.472961e-03
## + PCNM60        -5.591286e-03
## + PCNM22        -5.698249e-03
## + PCNM20        -8.945806e-03
## 
##          Df    AIC      F Pr(>F)   
## + PCNM37  1 261.25 2.3484  0.006 **
## ---
## Signif. codes:  0 '***' 0.001 '**' 0.01 '*' 0.05 '.' 0.1 ' ' 1
## 
## Step: R2.adj= 0.0146008 
## Call: woodytreedeamssm3.db ~ PCNM37 
##  
##                 R2.adjusted
## + PCNM34        0.029213604
## + PCNM52        0.028794590
## + PCNM43        0.027194529
## + PCNM44        0.026903848
## + PCNM56        0.024400766
## + PCNM39        0.023372632
## + PCNM40        0.023133271
## + PCNM51        0.020965514
## + PCNM45        0.020918527
## + PCNM38        0.020319526
## + PCNM46        0.020072552
## + PCNM29        0.020026619
## + PCNM53        0.019547989
## + PCNM55        0.019258724
## + PCNM42        0.018793128
## + PCNM26        0.018437059
## + PCNM58        0.018296816
## + PCNM35        0.018234076
## + PCNM13        0.018113034
## + PCNM7         0.017830495
## + PCNM36        0.017783361
## + PCNM24        0.017720583
## + PCNM33        0.016879764
## + PCNM4         0.016842338
## + PCNM54        0.016528042
## + PCNM18        0.016061538
## <All variables> 0.015750809
## + PCNM19        0.015395494
## + PCNM25        0.015385900
## + PCNM5         0.015377428
## + PCNM41        0.015321110
## + PCNM47        0.015169162
## + PCNM14        0.014958198
## + PCNM8         0.014943580
## + PCNM49        0.014610033
## <none>          0.014600795
## + PCNM32        0.014253036
## + PCNM30        0.014130498
## + PCNM28        0.014113848
## + PCNM61        0.013684475
## + PCNM21        0.012953479
## + PCNM17        0.012790617
## + PCNM15        0.012490994
## + PCNM2         0.012328773
## + PCNM57        0.012098163
## + PCNM48        0.011831303
## + PCNM23        0.011617123
## + PCNM12        0.011414245
## + PCNM9         0.011412591
## + PCNM10        0.011181216
## + PCNM31        0.011105012
## + PCNM59        0.010958159
## + PCNM11        0.010563430
## + PCNM16        0.010250713
## + PCNM3         0.009921665
## + PCNM50        0.009745287
## + PCNM1         0.009315714
## + PCNM6         0.009259286
## + PCNM27        0.008915036
## + PCNM22        0.008828865
## + PCNM60        0.008779499
## + PCNM20        0.005721943
\end{verbatim}

\begin{Shaded}
\begin{Highlighting}[]
\NormalTok{step.pcnm}\OperatorTok{$}\NormalTok{anova}
\end{Highlighting}
\end{Shaded}

\begin{verbatim}
##                   R2.adj Df    AIC      F Pr(>F)   
## + PCNM37        0.014601  1 261.25 2.3484  0.006 **
## <All variables> 0.015751                           
## ---
## Signif. codes:  0 '***' 0.001 '**' 0.01 '*' 0.05 '.' 0.1 ' ' 1
\end{verbatim}

We'll this is interesting. It looks like space doesn't actually
influence the presence of woody species that much. In the case of using
BC, the results are that the null model is chosen, in the case of
Sorensen's it is that only 1.4\% of variation is accounted for
spatially, with the model of PCNM37.

\begin{Shaded}
\begin{Highlighting}[]
\NormalTok{space.mod <-}\StringTok{ }\KeywordTok{model.matrix}\NormalTok{(}\OperatorTok{~}\NormalTok{PCNM37, }\KeywordTok{as.data.frame}\NormalTok{(woodytreedeamssm3.space))[,}\OperatorTok{-}\DecValTok{1}\NormalTok{]}
\NormalTok{farm.mod <-}\StringTok{ }\KeywordTok{model.matrix}\NormalTok{(}\OperatorTok{~}\NormalTok{pcImproved1920 }\OperatorTok{+}\StringTok{ }\NormalTok{ImprovedFarmAcres1920, }\KeywordTok{as.data.frame}\NormalTok{(popData))[,}\OperatorTok{-}\DecValTok{1}\NormalTok{]}
\end{Highlighting}
\end{Shaded}

\begin{Shaded}
\begin{Highlighting}[]
\NormalTok{woodytreedeamssm3.total.farm <-}\StringTok{ }\KeywordTok{dbrda}\NormalTok{(woodytreedeamssm3.db }\OperatorTok{~}\NormalTok{farm.mod)}
\end{Highlighting}
\end{Shaded}

\begin{Shaded}
\begin{Highlighting}[]
\NormalTok{woodytreedeamssm3.varpart <-}\StringTok{ }\KeywordTok{varpart}\NormalTok{(woodytreedeamssm3.db,farm.mod,space.mod)}
\NormalTok{woodytreedeamssm3.varpart}
\end{Highlighting}
\end{Shaded}

\begin{verbatim}
## 
## Partition of squared Binary bray distance in dbRDA 
## 
## Call: varpart(Y = woodytreedeamssm3.db, X = farm.mod, space.mod)
## 
## Explanatory tables:
## X1:  farm.mod
## X2:  space.mod 
## 
## No. of explanatory tables: 2 
## Total variation (SS): 16.994 
## No. of observations: 92 
## 
## Partition table:
##                      Df R.squared Adj.R.squared Testable
## [a+b] = X1            2   0.06070       0.03959     TRUE
## [b+c] = X2            1   0.02543       0.01460     TRUE
## [a+b+c] = X1+X2       3   0.08341       0.05217     TRUE
## Individual fractions                                    
## [a] = X1|X2           2                 0.03756     TRUE
## [b]                   0                 0.00202    FALSE
## [c] = X2|X1           1                 0.01258     TRUE
## [d] = Residuals                         0.94783    FALSE
## ---
## Use function 'dbrda' to test significance of fractions of interest
\end{verbatim}

\begin{Shaded}
\begin{Highlighting}[]
\KeywordTok{plot}\NormalTok{(woodytreedeamssm3.varpart)}
\end{Highlighting}
\end{Shaded}

\includegraphics{TeamProject5_files/figure-latex/unnamed-chunk-38-1.pdf}

\hypertarget{repeat-analysis-for-all-species}{%
\subsubsection{Repeat analysis for all
species}\label{repeat-analysis-for-all-species}}

\begin{Shaded}
\begin{Highlighting}[]
\NormalTok{compdatasitesp.db <-}\StringTok{ }\KeywordTok{vegdist}\NormalTok{(compdatasitesp,}\DataTypeTok{method=}\StringTok{'bray'}\NormalTok{,}\DataTypeTok{binary =} \OtherTok{TRUE}\NormalTok{,}\DataTypeTok{upper=}\OtherTok{TRUE}\NormalTok{,}\DataTypeTok{diag=}\OtherTok{TRUE}\NormalTok{)}

\NormalTok{compdatasitesp.dbrda <-}\StringTok{ }\KeywordTok{dbrda}\NormalTok{(compdatasitesp.db }\OperatorTok{~}\NormalTok{.,}\KeywordTok{as.data.frame}\NormalTok{(popData))}
\KeywordTok{ordiplot}\NormalTok{(compdatasitesp.dbrda)}
\end{Highlighting}
\end{Shaded}

\includegraphics{TeamProject5_files/figure-latex/unnamed-chunk-39-1.pdf}
Cool, from this it is clear that the census population levels are
minimally correlated, while the land area of counties and improved acres
are more correlated.

\begin{Shaded}
\begin{Highlighting}[]
\NormalTok{compdatasitesp.dbrda.mod0 <-}\KeywordTok{dbrda}\NormalTok{(compdatasitesp.db}\OperatorTok{~}\DecValTok{1}\NormalTok{,}\KeywordTok{as.data.frame}\NormalTok{(popData))}
\CommentTok{# ordiplot(compdatasitesp.dbrda.mod0)}
\end{Highlighting}
\end{Shaded}

\begin{Shaded}
\begin{Highlighting}[]
\NormalTok{compdatasitesp.dbrda.mod1 <-}\StringTok{ }\KeywordTok{dbrda}\NormalTok{(compdatasitesp.db}\OperatorTok{~}\NormalTok{., }\KeywordTok{as.data.frame}\NormalTok{(popData))}
\CommentTok{# ordiplot(compdatasitesp.dbrda.mod1)}
\end{Highlighting}
\end{Shaded}

\begin{Shaded}
\begin{Highlighting}[]
\NormalTok{compdatasitesp.dbrda <-}\StringTok{ }\KeywordTok{ordiR2step}\NormalTok{(compdatasitesp.dbrda.mod0,compdatasitesp.dbrda.mod1,}\DataTypeTok{perm.max=}\DecValTok{200}\NormalTok{)}
\end{Highlighting}
\end{Shaded}

\begin{verbatim}
## Step: R2.adj= 0 
## Call: compdatasitesp.db ~ 1 
##  
##                                R2.adjusted
## <All variables>               0.3151507393
## + EcoIV                       0.2302854482
## + EcoIII                      0.1952076711
## + `ctyspeciesscale$latitude`  0.1736210208
## + sampEffort                  0.1012821375
## + Glaciation                  0.0730063407
## + pcImproved1920              0.0412544203
## + ImprovedFarmAcres1920       0.0203855795
## + LandArea1920                0.0191080948
## + X1900                       0.0003941725
## <none>                        0.0000000000
## + popDensity1920             -0.0002290714
## + X2010                      -0.0008591065
## + X1910                      -0.0012692149
## + X2000                      -0.0014150218
## + X1990                      -0.0018920028
## + X1980                      -0.0021307636
## + X1920                      -0.0021639087
## + X1970                      -0.0024142605
## + X1960                      -0.0028397894
## + X1940                      -0.0029938178
## + X1950                      -0.0030254171
## + X1930                      -0.0031379804
## 
##         Df    AIC     F Pr(>F)   
## + EcoIV 15 262.36 2.815  0.002 **
## ---
## Signif. codes:  0 '***' 0.001 '**' 0.01 '*' 0.05 '.' 0.1 ' ' 1
## 
## Step: R2.adj= 0.2302854 
## Call: compdatasitesp.db ~ EcoIV 
##  
##                              R2.adjusted
## <All variables>                0.3151507
## + sampEffort                   0.2806322
## + `ctyspeciesscale$latitude`   0.2626344
## + EcoIII                       0.2539459
## + Glaciation                   0.2384513
## + LandArea1920                 0.2383871
## + pcImproved1920               0.2381939
## + ImprovedFarmAcres1920        0.2341728
## + popDensity1920               0.2307655
## <none>                         0.2302854
## + X1900                        0.2295894
## + X1910                        0.2294499
## + X2010                        0.2293265
## + X1920                        0.2292334
## + X2000                        0.2290489
## + X1990                        0.2288898
## + X1940                        0.2286835
## + X1980                        0.2286597
## + X1950                        0.2286202
## + X1930                        0.2286004
## + X1970                        0.2285730
## + X1960                        0.2284029
## 
##              Df    AIC      F Pr(>F)   
## + sampEffort  1 256.92 6.3191  0.002 **
## ---
## Signif. codes:  0 '***' 0.001 '**' 0.01 '*' 0.05 '.' 0.1 ' ' 1
## 
## Step: R2.adj= 0.2806322 
## Call: compdatasitesp.db ~ EcoIV + sampEffort 
##  
##                              R2.adjusted
## <All variables>                0.3151507
## + `ctyspeciesscale$latitude`   0.2989613
## + EcoIII                       0.2892776
## + pcImproved1920               0.2881481
## + Glaciation                   0.2880768
## + LandArea1920                 0.2826758
## + ImprovedFarmAcres1920        0.2811984
## + popDensity1920               0.2808578
## <none>                         0.2806322
## + X1900                        0.2805851
## + X1910                        0.2804491
## + X2010                        0.2803767
## + X1920                        0.2802533
## + X2000                        0.2801002
## + X1990                        0.2799451
## + X1980                        0.2797223
## + X1940                        0.2796786
## + X1970                        0.2796373
## + X1950                        0.2796263
## + X1930                        0.2795933
## + X1960                        0.2794489
## 
##                              Df    AIC      F Pr(>F)   
## + `ctyspeciesscale$latitude`  2 256.06 1.9805  0.002 **
## ---
## Signif. codes:  0 '***' 0.001 '**' 0.01 '*' 0.05 '.' 0.1 ' ' 1
## 
## Step: R2.adj= 0.2989613 
## Call: compdatasitesp.db ~ EcoIV + sampEffort + `ctyspeciesscale$latitude` 
##  
##                         R2.adjusted
## <All variables>           0.3151507
## + pcImproved1920          0.3079486
## + Glaciation              0.3021881
## + LandArea1920            0.3015751
## + ImprovedFarmAcres1920   0.2997365
## + EcoIII                  0.2997113
## + popDensity1920          0.2996937
## + X2010                   0.2995529
## + X1900                   0.2994155
## + X1910                   0.2993246
## + X2000                   0.2992344
## + X1920                   0.2991353
## + X1990                   0.2990172
## <none>                    0.2989613
## + X1980                   0.2987402
## + X1970                   0.2985846
## + X1940                   0.2985824
## + X1950                   0.2985372
## + X1930                   0.2985190
## + X1960                   0.2983486
## 
##                  Df   AIC     F Pr(>F)   
## + pcImproved1920  1 255.6 1.948  0.002 **
## ---
## Signif. codes:  0 '***' 0.001 '**' 0.01 '*' 0.05 '.' 0.1 ' ' 1
## 
## Step: R2.adj= 0.3079486 
## Call: compdatasitesp.db ~ EcoIV + sampEffort + `ctyspeciesscale$latitude` +      pcImproved1920 
##  
##                         R2.adjusted
## <All variables>           0.3151507
## + ImprovedFarmAcres1920   0.3109468
## + LandArea1920            0.3109073
## + popDensity1920          0.3092621
## + EcoIII                  0.3089603
## + X2010                   0.3088300
## + X1910                   0.3086417
## + X1900                   0.3086148
## + X1920                   0.3085693
## + X2000                   0.3085426
## + X1990                   0.3083677
## + X1980                   0.3081214
## + X1940                   0.3080635
## + X1970                   0.3080257
## + X1950                   0.3080195
## + X1930                   0.3079875
## <none>                    0.3079486
## + X1960                   0.3078328
## + Glaciation              0.3068888
## 
##                         Df    AIC      F Pr(>F)  
## + ImprovedFarmAcres1920  1 255.91 1.3133   0.05 *
## ---
## Signif. codes:  0 '***' 0.001 '**' 0.01 '*' 0.05 '.' 0.1 ' ' 1
## 
## Step: R2.adj= 0.3109468 
## Call: compdatasitesp.db ~ EcoIV + sampEffort + `ctyspeciesscale$latitude` +      pcImproved1920 + ImprovedFarmAcres1920 
##  
##                  R2.adjusted
## <All variables>    0.3151507
## + X2010            0.3125910
## + X2000            0.3123465
## + popDensity1920   0.3123307
## + X1910            0.3122844
## + X1920            0.3122773
## + X1990            0.3121901
## + X1900            0.3120710
## + X1980            0.3120077
## + EcoIII           0.3119150
## + X1970            0.3118587
## + X1940            0.3118004
## + X1930            0.3117607
## + X1950            0.3117278
## + X1960            0.3116260
## <none>             0.3109468
## + LandArea1920     0.3105628
## + Glaciation       0.3093596
## 
##         Df    AIC      F Pr(>F)
## + X2010  1 256.39 1.1698   0.26
\end{verbatim}

\begin{Shaded}
\begin{Highlighting}[]
\NormalTok{compdatasitesp.dbrda}\OperatorTok{$}\NormalTok{anova}
\end{Highlighting}
\end{Shaded}

\begin{verbatim}
##                               R2.adj Df    AIC      F Pr(>F)   
## + EcoIV                      0.23028 15 262.36 2.8150  0.002 **
## + sampEffort                 0.28063  1 256.92 6.3191  0.002 **
## + `ctyspeciesscale$latitude` 0.29896  2 256.06 1.9805  0.002 **
## + pcImproved1920             0.30795  1 255.60 1.9480  0.002 **
## + ImprovedFarmAcres1920      0.31095  1 255.91 1.3133  0.050 * 
## <All variables>              0.31515                           
## ---
## Signif. codes:  0 '***' 0.001 '**' 0.01 '*' 0.05 '.' 0.1 ' ' 1
\end{verbatim}

\begin{Shaded}
\begin{Highlighting}[]
\KeywordTok{ordiplot}\NormalTok{(compdatasitesp.dbrda)}
\end{Highlighting}
\end{Shaded}

\includegraphics{TeamProject5_files/figure-latex/unnamed-chunk-42-1.pdf}

This is actually better with a 10\% from these two variables

\begin{Shaded}
\begin{Highlighting}[]
\NormalTok{rs <-}\StringTok{ }\KeywordTok{rowSums}\NormalTok{(compdatasitesp)}\OperatorTok{/}\KeywordTok{sum}\NormalTok{(compdatasitesp)}
\NormalTok{compdatasitesp.pcnmw <-}\StringTok{ }\KeywordTok{pcnm}\NormalTok{(}\KeywordTok{dist}\NormalTok{(indcounties4}\OperatorTok{$}\NormalTok{xy),}\DataTypeTok{w=}\NormalTok{rs,}\DataTypeTok{dist.ret =} \OtherTok{TRUE}\NormalTok{)}

\NormalTok{compdatasitesp.space <-}\KeywordTok{as.data.frame}\NormalTok{(}\KeywordTok{scores}\NormalTok{(compdatasitesp.pcnmw))}
\NormalTok{compdatasitesp.pcnm.mod0 <-}\KeywordTok{dbrda}\NormalTok{(compdatasitesp.db }\OperatorTok{~}\DecValTok{1}\NormalTok{,compdatasitesp.space)}
\NormalTok{compdatasitesp.pcnm.mod1 <-}\KeywordTok{dbrda}\NormalTok{(compdatasitesp.db }\OperatorTok{~}\NormalTok{.,compdatasitesp.space)}
\NormalTok{step.pcnm <-}\StringTok{ }\KeywordTok{ordiR2step}\NormalTok{(compdatasitesp.pcnm.mod0, compdatasitesp.pcnm.mod1,}\DataTypeTok{perm.max=}\DecValTok{500}\NormalTok{)}
\end{Highlighting}
\end{Shaded}

\begin{verbatim}
## Step: R2.adj= 0 
## Call: compdatasitesp.db ~ 1 
##  
##                   R2.adjusted
## <All variables>  2.345528e-02
## + PCNM53         1.237248e-02
## + PCNM7          1.217106e-02
## + PCNM52         1.133330e-02
## + PCNM40         8.595785e-03
## + PCNM50         6.620612e-03
## + PCNM45         5.248071e-03
## + PCNM29         5.073930e-03
## + PCNM33         4.849425e-03
## + PCNM46         4.462560e-03
## + PCNM20         3.411027e-03
## + PCNM37         3.272998e-03
## + PCNM42         2.815825e-03
## + PCNM41         2.499659e-03
## + PCNM2          2.489840e-03
## + PCNM55         2.423176e-03
## + PCNM49         2.255920e-03
## + PCNM57         2.194791e-03
## + PCNM43         2.167195e-03
## + PCNM31         2.100845e-03
## + PCNM44         2.034557e-03
## + PCNM27         1.570521e-03
## + PCNM38         1.508059e-03
## + PCNM59         1.465510e-03
## + PCNM10         1.438199e-03
## + PCNM19         1.235930e-03
## + PCNM54         1.050425e-03
## + PCNM34         8.989576e-04
## + PCNM25         6.898663e-04
## + PCNM47         3.125497e-04
## + PCNM39         3.111183e-04
## <none>           0.000000e+00
## + PCNM61        -8.317531e-05
## + PCNM23        -1.140570e-04
## + PCNM28        -1.976430e-04
## + PCNM18        -2.444915e-04
## + PCNM56        -4.300297e-04
## + PCNM48        -4.697309e-04
## + PCNM30        -5.205693e-04
## + PCNM32        -6.082432e-04
## + PCNM35        -6.170726e-04
## + PCNM51        -6.654117e-04
## + PCNM17        -1.188689e-03
## + PCNM26        -1.292155e-03
## + PCNM4         -1.538095e-03
## + PCNM36        -1.605302e-03
## + PCNM24        -1.850222e-03
## + PCNM12        -1.968083e-03
## + PCNM13        -2.243513e-03
## + PCNM15        -2.428115e-03
## + PCNM16        -2.557032e-03
## + PCNM58        -2.835485e-03
## + PCNM11        -2.868988e-03
## + PCNM1         -3.062991e-03
## + PCNM60        -3.279015e-03
## + PCNM6         -3.332669e-03
## + PCNM21        -3.395564e-03
## + PCNM22        -3.640774e-03
## + PCNM9         -4.018879e-03
## + PCNM8         -4.370016e-03
## + PCNM5         -4.782621e-03
## + PCNM14        -4.821995e-03
## + PCNM3         -5.139794e-03
## 
##          Df    AIC    F Pr(>F)   
## + PCNM53  1 272.85 2.14  0.002 **
## ---
## Signif. codes:  0 '***' 0.001 '**' 0.01 '*' 0.05 '.' 0.1 ' ' 1
## 
## Step: R2.adj= 0.01237248 
## Call: compdatasitesp.db ~ PCNM53 
##  
##                 R2.adjusted
## + PCNM7         0.025271742
## + PCNM52        0.024979067
## <All variables> 0.023455279
## + PCNM40        0.019748276
## + PCNM50        0.019584894
## + PCNM29        0.017693227
## + PCNM45        0.017176138
## + PCNM46        0.017164231
## + PCNM33        0.016914427
## + PCNM20        0.016066808
## + PCNM42        0.015364928
## + PCNM37        0.015131728
## + PCNM49        0.014813034
## + PCNM41        0.014805656
## + PCNM57        0.014745575
## + PCNM43        0.014711604
## + PCNM2         0.014528370
## + PCNM31        0.014469921
## + PCNM59        0.014084489
## + PCNM19        0.014043553
## + PCNM38        0.013994403
## + PCNM44        0.013994250
## + PCNM27        0.013919297
## + PCNM55        0.013878204
## + PCNM54        0.013575460
## + PCNM25        0.013447943
## + PCNM34        0.013416574
## + PCNM10        0.013075877
## + PCNM39        0.012732636
## + PCNM47        0.012689556
## + PCNM23        0.012579518
## + PCNM28        0.012394978
## <none>          0.012372477
## + PCNM61        0.012336621
## + PCNM18        0.012278866
## + PCNM56        0.012041101
## + PCNM48        0.012016830
## + PCNM35        0.012009882
## + PCNM30        0.011737262
## + PCNM32        0.011725397
## + PCNM26        0.011451000
## + PCNM17        0.011260488
## + PCNM51        0.011233284
## + PCNM4         0.010989734
## + PCNM36        0.010837356
## + PCNM12        0.010824157
## + PCNM24        0.010620070
## + PCNM13        0.009961452
## + PCNM16        0.009921389
## + PCNM15        0.009919836
## + PCNM58        0.009780332
## + PCNM11        0.009609459
## + PCNM1         0.009407457
## + PCNM60        0.009203686
## + PCNM6         0.009201212
## + PCNM21        0.008966865
## + PCNM22        0.008877135
## + PCNM9         0.008333733
## + PCNM8         0.008036499
## + PCNM5         0.007726867
## + PCNM14        0.007641260
## + PCNM3         0.007337265
\end{verbatim}

\begin{Shaded}
\begin{Highlighting}[]
\NormalTok{step.pcnm}\OperatorTok{$}\NormalTok{anova}
\end{Highlighting}
\end{Shaded}

\begin{verbatim}
##                   R2.adj Df    AIC    F Pr(>F)   
## + PCNM53        0.012372  1 272.85 2.14  0.002 **
## <All variables> 0.023455                         
## ---
## Signif. codes:  0 '***' 0.001 '**' 0.01 '*' 0.05 '.' 0.1 ' ' 1
\end{verbatim}

\#need to change

\begin{Shaded}
\begin{Highlighting}[]
\NormalTok{space.mod <-}\StringTok{ }\KeywordTok{model.matrix}\NormalTok{(}\OperatorTok{~}\NormalTok{PCNM53, }\KeywordTok{as.data.frame}\NormalTok{(compdatasitesp.space))[,}\OperatorTok{-}\DecValTok{1}\NormalTok{]}
\NormalTok{farm.mod <-}\StringTok{ }\KeywordTok{model.matrix}\NormalTok{(}\OperatorTok{~}\NormalTok{pcImproved1920 }\OperatorTok{+}\StringTok{ }\NormalTok{ImprovedFarmAcres1920, }\KeywordTok{as.data.frame}\NormalTok{(popData))[,}\OperatorTok{-}\DecValTok{1}\NormalTok{]}
\end{Highlighting}
\end{Shaded}

\begin{Shaded}
\begin{Highlighting}[]
\NormalTok{compdatasitesp.total.farm <-}\StringTok{ }\KeywordTok{dbrda}\NormalTok{(compdatasitesp.db }\OperatorTok{~}\NormalTok{farm.mod)}
\end{Highlighting}
\end{Shaded}

\begin{Shaded}
\begin{Highlighting}[]
\NormalTok{compdatasitesp.varpart <-}\StringTok{ }\KeywordTok{varpart}\NormalTok{(compdatasitesp.db,farm.mod,space.mod)}
\NormalTok{compdatasitesp.varpart}
\end{Highlighting}
\end{Shaded}

\begin{verbatim}
## 
## Partition of squared Binary bray distance in dbRDA 
## 
## Call: varpart(Y = compdatasitesp.db, X = farm.mod, space.mod)
## 
## Explanatory tables:
## X1:  farm.mod
## X2:  space.mod 
## 
## No. of explanatory tables: 2 
## Total variation (SS): 19.235 
## No. of observations: 92 
## 
## Partition table:
##                      Df R.squared Adj.R.squared Testable
## [a+b] = X1            2   0.08305       0.06245     TRUE
## [b+c] = X2            1   0.02323       0.01237     TRUE
## [a+b+c] = X1+X2       3   0.10188       0.07126     TRUE
## Individual fractions                                    
## [a] = X1|X2           2                 0.05889     TRUE
## [b]                   0                 0.00356    FALSE
## [c] = X2|X1           1                 0.00882     TRUE
## [d] = Residuals                         0.92874    FALSE
## ---
## Use function 'dbrda' to test significance of fractions of interest
\end{verbatim}

\begin{Shaded}
\begin{Highlighting}[]
\KeywordTok{plot}\NormalTok{(compdatasitesp.varpart)}
\end{Highlighting}
\end{Shaded}

\includegraphics{TeamProject5_files/figure-latex/unnamed-chunk-47-1.pdf}

\begin{Shaded}
\begin{Highlighting}[]
\NormalTok{popData}\OperatorTok{$}\NormalTok{county <-}\StringTok{ }\KeywordTok{rownames}\NormalTok{(popData)}
\NormalTok{popData[}\StringTok{"La Porte"}\NormalTok{,}\StringTok{"county"}\NormalTok{]=}\StringTok{"LaPorte"}
\NormalTok{map <-}\StringTok{ }\KeywordTok{merge}\NormalTok{(indcounties3,}
\NormalTok{                     popData,}
                     \DataTypeTok{by.x =} \StringTok{"county"}\NormalTok{,}
                     \DataTypeTok{by.y =} \StringTok{"county"}\NormalTok{)}
\CommentTok{# Plot}
\KeywordTok{tm_shape}\NormalTok{(map, }\DataTypeTok{projection=}\StringTok{"+init=epsg:4326"}\NormalTok{) }\OperatorTok{+}
\StringTok{  }\KeywordTok{tm_polygons}\NormalTok{(}\StringTok{"pcImproved1920"}\NormalTok{, }\DataTypeTok{border.col =} \StringTok{"grey30"}\NormalTok{, }\DataTypeTok{title=}\StringTok{""}\NormalTok{) }\OperatorTok{+}
\KeywordTok{tm_layout}\NormalTok{(}\DataTypeTok{title=}\StringTok{"Percent Improved Agricultural Land"}\NormalTok{, }
          \DataTypeTok{title.size =} \DecValTok{1}\NormalTok{, }
          \DataTypeTok{title.position =} \KeywordTok{c}\NormalTok{(}\StringTok{"center"}\NormalTok{, }\StringTok{"top"}\NormalTok{), }
          \DataTypeTok{legend.text.size=}\NormalTok{.}\DecValTok{6}\NormalTok{, }\DataTypeTok{bg.color =} \StringTok{"light blue"}\NormalTok{, }\DataTypeTok{inner.margins=}\KeywordTok{c}\NormalTok{(.}\DecValTok{1}\NormalTok{,.}\DecValTok{1}\NormalTok{,.}\DecValTok{2}\NormalTok{,.}\DecValTok{4}\NormalTok{))}
\end{Highlighting}
\end{Shaded}

\includegraphics{TeamProject5_files/figure-latex/unnamed-chunk-48-1.pdf}

\begin{Shaded}
\begin{Highlighting}[]
\KeywordTok{tm_legend}\NormalTok{(}\DataTypeTok{position =} \KeywordTok{c}\NormalTok{(}\StringTok{"right"}\NormalTok{, }\StringTok{"bottom"}\NormalTok{), }
    \DataTypeTok{frame =} \OtherTok{TRUE}\NormalTok{,}
    \DataTypeTok{bg.color=}\StringTok{"lightblue"}\NormalTok{)}
\end{Highlighting}
\end{Shaded}

\begin{verbatim}
## $tm_layout
## $tm_layout$legend.position
## [1] "right"  "bottom"
## 
## $tm_layout$legend.frame
## [1] TRUE
## 
## $tm_layout$legend.bg.color
## [1] "lightblue"
## 
## $tm_layout$style
## [1] NA
## 
## 
## attr(,"class")
## [1] "tm"
\end{verbatim}

\end{document}
